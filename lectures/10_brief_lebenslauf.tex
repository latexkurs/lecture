% !TEX program = xelatex
% !TEX encoding = UTF-8 Unicode
% !TEX spellcheck = de_DE
% 
% © 2015–2017 Moritz Brinkmann, CC-by-sa
% © 2018–2022 Maximilian Jalea, CC-by-sa
% http://latexkurs.github.io

\documentclass[
	vorläufig=false,
	datum=2022-12-21,
	titel={Briefe und Lebensläufe},
	web=true,
%	noshortverb=true,
	max,
	aspectratio=1610
]{../tex/latexkurs-slides}


\usepackage{
	array,
	booktabs,
}

\begin{document}

\begin{frame}{Übersicht}
	\tableofcontents
\end{frame}


\section{Briefe}
\frame{\centering\alert{Teil I}\\\huge Briefe}

\subsection{Briefe mit KOMA-Script: \texttt{scrlettr2}}

\begin{frame}{scrlettr2: Grundidee}
    \begin{itemize}
		\item wie immer in \LaTeX: Trennung von Form und Inhalt
		\item alle formalen Elemente werden per Makro gesetzt
		\item Briefinhalt selbst wird direkt eingegeben
		\item Positionierung von Elementen mittels Befehlen anpassbar
	\end{itemize}
\end{frame}

\begin{frame}[fragile]{scrlettr2: neue Syntax}
\LaTeX\ kennt folgende „Dinge“:
    \begin{itemize}
		\item Befehle (|\texttt{}|)
		\item Umgebungen (|\begin{abstract} \end{abstract}|)
		\item Zähler (|\thepage|)
		\item Längen (|\pageheight=3cm|)
		\item Optionen (einfacher Wert oder Key-Value: |ngerman|, |top=2cm|)
	\end{itemize}
KOMA-Skript erweitert dies um:
    \begin{itemize}
		\item Elemente (|\setkomafont{title}{\fontspec{Arno Pro}}|) \pause
		\item \alert{Variablen} (nur in der Briefklasse \pkg{scrlettr2})
	\end{itemize}
\end{frame}

\begin{frame}[fragile]{Variablen in scrlettr2}
    \begin{itemize}
		\item Setzen von Variablen mittels |\setkomavar{|\meta{Variable}|}{|\meta{Wert}|}|
		\item \alert{nicht zu verwechseln} mit |\KOMAoptions{}|
		\item mögliche Elemente: (kleine Auswahl)
	\end{itemize}

\begin{tabular}{>{\ttfamily}ll}\toprule
fromname & Absendername\\
fromaddress & Absenderadresse\\
fromemail & E-Mailadresse des Absenders\\\pause
myref & Feld für „Mein Zeichen“\\
specialmail & Versandart (Luftpost …)\\
backaddressseparator & Trennzeichen in der Rücksendeadresse\\\bottomrule
\end{tabular}\\[2ex]
⇒ siehe \href{http://mirrors.ctan.org/macros/latex/contrib/koma-script/doc/scrguide.pdf}{\texttt{texdoc scrguide}}
\end{frame}

\begin{frame}[fragile]{Setzen von Variablen}
    \begin{itemize}
		\item Variablen verfügen über \alert{Inhalt}:\\%
\begin{lstlisting}
\setkomavar{fromname}{Mustermann}
\end{lstlisting}
		\item aber auch über \alert{Bezeichnung}:\\%
\begin{lstlisting}
\setkomavar*{fromname}{Absender}  % statt: Von
\end{lstlisting}
		\item Kurzform:\\%
\begin{lstlisting}
\setkomavar{fromname}[Absender]{Musterfrau}
\end{lstlisting}
		\item Ausgabe:\\%
Absender: Musterfrau
	\end{itemize}
\end{frame}

\begin{frame}[fragile]{Nutzen von Variablen}
	\begin{itemize}
		\item normalerweise werden Variablen nur gesetzt und von der Klasse genutzt
		\item Dokumentklasse kümmert sich dann um alles
		\item eigene Variablen können definiert werden
		\item Verwendung mittels |\usekomavar|
	\end{itemize}
	\vfill
|\newkomavar[|\meta{Bezeichnung}|]{|\meta{Name}|}|\\[1ex]
|\usekomavar[|\meta{Formatierung}|]{fromname}| ⇒ |Musterfrau|\\
|\usekomavar*[|\meta{Formatierung}|]{fromname}| ⇒ |Absender|\\[1ex]
Dabei kann mit \meta{Formatierung} beliebiger Code ausgeführt werden\\
(z.\,B. |\bfseries|, |\MakeUppercase|)
\end{frame}

\begin{frame}[fragile]{Beispiel}
\begin{olcol}
\begin{lstlisting}
\documentclass{scrlttr2}

\setkomavar{fromname}{Maximilian Jalea}
\setkomavar{fromaddress}{Im Neuenheimer Feld 205\\69120 Heidelberg}

\begin{document}

  \begin{letter}{Prof. Dr. Dr. h.c. Bernhard Eitel\\Grabengasse 1\\69117 Heidelberg}
    \opening{Sehr geehrter Herr Rektor,}
      dies ist mein erster Brief.
    \closing{Gruß}
  \end{letter}
  
\end{document}
\end{lstlisting}
\end{olcol}
\end{frame}

\begin{frame}[fragile]{Besonderheiten}
    \begin{itemize}
		\item \pkg{scrlettr2} unterscheidet sich in der Bedienung von anderen Klassen:
		\item es werden erst Briefe gesetzt, wenn |\opening{}| angegeben wird! % was wiederum nur in |letter|-Umgebung vorkommen darf …
		\item nur sehr wenige Elemente werden dort angegeben, wo sie verwendet werden
		\item[⇒] sehr strikte Trennung von Form und Inhalt
	\end{itemize}
\end{frame}

\subsection{Letter Class Options}
\begin{frame}[fragile]{letter class option}
    \begin{itemize}
		\item Für standardisiertes Layout: immer gleiche Einstellungen
		\item[\alert{⇒}] copy \& paste?
		\item[\alert{⇒}] eigene |.cls| oder |.sty|-Datei?
		\item[\alert{⇒}] eigene |.tex|?
		\item[\alert{⇒}] Inkompatibilität, nicht gut portierbar\pause
		\item[⇒] eigenes Format für \pkg{scrlettr2}: |.lco|-Dateien
	\end{itemize}
\end{frame}

\begin{frame}[fragile]{letter class option}
    \begin{itemize}
		\item KOMA definiert bereits einige |.lco|-Dateien
		\item einfache Definition eigener |.lco|
		\item leichter Austausch\\%
⇒ normierte Geschäftsbriefe möglich
		\item nach Laden Anpassungen möglich\\%
⇒ dem Zweck angepasstes, schönes Format
		\item Verwendung: Als Klassenoption:\\%
|\documentclass[|\meta{lco-Name}|]{scrlttr2}|\\%
oder im Dokument\\%
|\LoadLetterOption{|\meta{lco-Name}|}|
	\end{itemize}
\end{frame}

\begin{frame}{letter class option}
\begin{table}
\begin{tabular}{ll}\toprule
DIN & gemäß DIN 676\\
DINmtext & Alternative für mehr Text auf der ersten Seite\\
KOMAold & Aussehen der alten \pkg{scrlettr}-Klasse\\
NipponEL & japanische Briefe\\
NipponEH & alternative japanische Briefe\\
SN & schweizer Briefe nach SN 010 130 (Anschrift rechts)\\
SNleft & dito, Anschrift links\\\bottomrule
\end{tabular}
\caption{einige Voreinstellungen für lco-Dateien}
\end{table}
Erstellen eigener |.lco|: siehe Dokumentation
\end{frame}

\subsection{Adressverwaltung}
\begin{frame}[fragile]{Adressverwaltung}
    \begin{itemize}
		\item Eingabe von Adressen nervig, zeitaufwändig und fehleranfällig
		\item Widerspricht dem Ansatz von \LaTeX
		\item[⇒] |.adr|-Dateien verwalten Adressen
		\item Einträge mit |\adrentry| bzw. |\addrentry|
		\item Verwenden mit |\input{adressen.adr}|
	\end{itemize}
\end{frame}

\begin{frame}[fragile]{adrentry vs. addrentry}
    \begin{itemize}
		\item |\adrentry| nimmt 8 Argumente
		\item |\addrentry| nimmt 9 Argumente
		\item letztes Argument definiert Befehl |\Kürzel|
	\end{itemize}
\begin{columns}
\begin{column}{.1\textwidth}
\begin{verbatim}
\adrentry{Name}
  {Vorname}
  {Adresse}
  {Telefon}
  {frei1}
  {frei2}
  {Kommentar}
  {Kürzel}
\end{verbatim}
\end{column}
\begin{column}{.1\textwidth}
\begin{verbatim}
\addrentry{Name}
  {Vorname}
  {Adresse}
  {Telefon}
  {frei1}
  {frei2}
  {frei3}
  {frei4}
  {Kürzel}
\end{verbatim}
\end{column}
\end{columns}
\end{frame}

\begin{frame}[fragile]{automatische Adressen}
    \begin{itemize}
		\item Verwendung im Brief:
	\end{itemize}
\begin{verbatim}
\begin{letter}{\Kürzel}
\opening{...}
\end{letter}
\end{verbatim}
⇒ Setzt automatisch die Adresse, die zum Eintrag |Kürzel| gehört\\
(z.\,B. |\MJalea|)
\end{frame}

\begin{frame}[fragile]{adrconv}
    \begin{itemize}
		\item damit die ganze Arbeit nicht nur im Brief steht:
		\item Paket \pkg{adrconv} kann Adressverzeichnisse oder Telefonlisten erstellen
		\item verwendet |\adrentry|, |\adrchar{E}| (wird von \pkg{scrlttr2} ignoriert)\\%
oder eigene Datenbank
		\item[⇒] |texdoc adrconv|
	\end{itemize}
\end{frame}

\subsection{Serienbriefe}
\begin{frame}[fragile]{Serienbriefe}
    \begin{itemize}
		\item „Missbrauch“ der Adressdatei:
		\item umdefinieren von |\ad(d)rentry| als Briefanfang
		\item[⇒] erstellt Brief an alle Einträge
	\end{itemize}
\pause
\begin{lstlisting}
\renewcommand{\adrentry}[8]{%
  \begin{letter}{#2 #1\\#3}
    \opening{Sehr geehrte Geschäftsparnter,}
    die nächste Sitzung findet morgen statt!
    \closing{Hochachtungsvoll}
  \end{letter}
}
\input{geschäftspartner.adr}
\end{lstlisting}
\end{frame}

\section{Lebensläufe}
\frame{\centering\alert{Teil II}\\\huge Lebensläufe}


\begin{frame}{Lebensläufe}
    \begin{itemize}
		\item professionelles Layout für Bewerbungen
		\item häufig standardisiert
		\item schlichtes Layout besser als überladenes
		\item Farben dezent einsetzen!
		\item Layout dem Zweck anpassen\\%
(Wohnheim, Universität, Bestattungsinstitut, \dots)
	\end{itemize}
\end{frame}

\subsection[europecv]{europecv/europasscv – europäische Standards}
\begin{frame}[fragile]{\href{http://ctan.org/pkg/europecv}{europecv}}
\begin{fancyquote}
\noindent As of 11 March 2002 the European Commission has defined a common format for curricula vitæ. This
class is an unofficial \LaTeX\ implementation of that format. Although primarily intended for users in the
European Union, the class can be used for any kind of curriculum vitæ.
\end{fancyquote}
    \begin{itemize}
		\item gute Dokumentation
		\item schlichtes, „klassisches“ Layout
		\item ausreichend formatierbar
	\end{itemize}
	\begin{itemize}
		\item Neues „offizielles“ Layout: \pkg{europasscv}
	\end{itemize}
\end{frame}

\subsection{moderncv}
\begin{frame}[fragile]{\href{http://ctan.org/pkg/moderncv}{moderncv}}
    \begin{itemize}
		\item bietet ein modernes, lockeres Layout
		\item \alert{keine} offizielle Dokumentation
		\item[⇒] Beispieldokumente, README (|texdoc -s moderncv|)
		\item[⇒] |moderncv.cls| ansehen
	\end{itemize}
\end{frame}

\subsection{curve – Trennung in Rubrikdateien}
\begin{frame}[fragile]{\href{http://ctan.org/pkg/curve}{curve}}
    \begin{itemize}
		\item Grundidee: Trennung von Hauptdokument (skeleton) und Inhalt
		\item Inhalte (Rubriken) stehen in eigenen Datein
		\item unterschiedliche |\flavor| möglich: \\%
je nach Zweck angepasster Lebenslauf
		\item Dateinamen: |name.flavorname.rubrikname|:\\%
|sprachkenntnisse.mpi.tex|\\%
|programmierkenntnisse.mpi.tex|\\%
|grogrammierkenntnisse.dante.tex|
		\item Einbinden mittels |\makerubric{dateiname}|
	\end{itemize}
\end{frame}

\subsection{simplecv – kiss}
\begin{frame}[fragile]{\href{http://ctan.org/pkg/simplecv}{simplecv}}
    \begin{itemize}
		\item einfacher und schichter, schnell zu erzeugender Lebenslauf
		\item Setzen von Headern: |\leftheader{}\rightheader{}|
		\item |\title|, |\maketitle| wie gewohnt
		\item |\section| und |\subsection| zur Strukturierung
		\item Aufzählungen in der |topic|-Umgebung
		\item Bibliographie möglich!
		\item Dokumentation am einfachsten über Suchfunktion von \pkg{texdoc}\\ (u.\,U. selbst kompilieren)
	\end{itemize} 
\end{frame}

\subsection{cv4tw}
\begin{frame}[fragile]{\href{http://ctan.org/pkg/cv4tw}{cv4tw}}
    \begin{itemize}
		\item Lebenslauf-Klasse für die Web-2.0-Generation
		\item Vielzahl von Social-Media-Icons
		\item Skill-Level in $n$ von 5 Sternen
		\item keine richtige Dokumentation, aber nette Beispiele (|texdoc -s cv4tw|)
	\end{itemize} 
\end{frame}

\nocite{dante:koma, moderncv, europasscv}
\begin{frame}[allowframebreaks]{Weiterführende Literatur}
\printbibliography
\end{frame}

\begin{frame}{\dantelogo}
\href{http://www.dante.de/index/Intern/Mitglied.html}{\dantelogo: Deutschsprachige Anwenderverenigung \TeX\ e.\,V.}\\
\begin{itemize}
	\item Fördert die Weiterentwicklung von \hologo{LaTeXTeX}
	\item betreibt den deutschen CTAN-Knoten
	\item zwei jährliche Konferenzen in wechselnden Städten
	\item Mitgliederzeitschrift: \emph{Die \TeX nische Komödie}
	\item Mitgliedsbeitrag für Studis: 20\,€\,a$^{-1}$
\end{itemize}
\end{frame}

\end{document}
