% !TEX program = xelatex
% !TEX encoding = UTF-8 Unicode
% !TEX spellcheck = de_DE
% 
% © 2015–2017 Moritz Brinkmann, CC-by-sa
% http://latexkurs.github.io

\documentclass[
	vorläufig=false,
	datum=2018-11-12,
	titel={Mathematiksatz II},
	web=true,
%	noshortverb=true,
	max,
	aspectratio=1610,
]{../tex/latexkurs-slides}

\usepackage{
	mathtools,
	12many,
	amsthm,
	braket,
	relsize,
%	breqn,
	cases,
	esvect,
	gnuplottex,
	soul,
	ulem,
	dsfont,
	siunitx,
	xfrac,
}
\normalem

\begin{document}


\begin{frame}{Übersicht}
	\tableofcontents
\end{frame}



\section{Mathe}
\teil{Mathe}
\subsection{Relationen}
\begin{frame}[fragile]{Relationen}
\begin{LTXexample}
\(= \equiv \approx \asymp \bowtie \cong \dashv \doteq \sim \simeq \propto \smile\)
\end{LTXexample}
\pause Negierung mit |\not|

\begin{LTXexample} 
\(\not = \neq \not\equiv
\not \approx \not A \not\kern-.25em A
\not\kern-.2em\int \not\kern-.2em\partial \not \smile\)
\end{LTXexample}
\pause Stapeln von Symbolen

\begin{LTXexample}
\(\stackrel{oben}{unten}\)
\(\stackrel{\text e}{\text a}=\)ä
\(\stackrel . = \neq \doteq\)
\(\stackrel != \stackrel ?=\)
\end{LTXexample}
\end{frame}


\subsection{Mengen}
\begin{frame}[fragile]{12many}
\begin{itemize}
\item Paket \pkg{12many} bietet Vereinfachung und Anpassung zum Mengensatz:\\%
\(\{1, \dots, m\}\)
\item Befehle:\\
|\nto{n}{m}|, |\ito{m}|, |\oto{m}|
\item Stil ändern mit |\setOTMstyle[]{|\meta{style}|}|
\end{itemize}
\begin{LTXexample}[width=.6\textwidth]
\( \nto{i}{k},
  \ito{m},
  \oto{\alpha_i} \)
\end{LTXexample}
\end{frame}


\subsection{Integrale}
\begin{frame}[fragile]{Integrale}
\AmS{}math bietet weitere Integrale:
\begin{LTXexample}[width=.45\textwidth]
\[ \iint \iiint \iiiint \]
\[ \oint \idotsint \]
\[ \int \int \]
\end{LTXexample}
\end{frame}

\begin{frame}[fragile]{Integrale}
\begin{columns}
\begin{column}{.4\textwidth}
Zusätzliche Integraldarstellungen bieten:
\begin{itemize}
\item \pkg{wasysym}
\item \pkg{txfonts}
\item \pkg{esint}
\item \pkg{MnSymbol}
\item \pkg{mathdesign}
\end{itemize}
\end{column}
\begin{column}{.5\textwidth}
\alert{Auf Kompatibilität achten}\\
Verschiedene Matheschriften zusammen können Probleme bereiten.
\end{column}
\end{columns}
\end{frame}


\subsection{Komplexe Matrizen}
\begin{frame}[fragile]{Satz komplexer Matrizen}
\begin{LTXexample}[width=.4\textwidth]
\[
\begin{pmatrix}
  a & b     & \dots & z\\
  b & \dots & \dots & z\\
  \vdots & \ddots& \reflectbox{\(\ddots\)} 
                              & \vdots\\
  \hdotsfor{4}\\
  z & b & \dots &
      \begin{pmatrix}
        a & b \\ c & d
      \end{pmatrix}
\end{pmatrix}
\]

\end{LTXexample}
\end{frame}


\subsection{Typische Mathe-Umgebungen}
\begin{frame}[fragile]{Typische Mathe-Umgebungen}%{Definition, Satz, Beweis, Lemma, …}
Mit dem \AmS-Paket \pkg{amsthm} lassen sich typische Mathe-Umgebungen wie „Satz“ und „Beweis“ erstellen:
\begin{itemize}
\item Anlegen einer Umgebungen mit |\newtheorem{|\meta{Kürzel}|}{|\meta{Name}|}[|\meta{Nummerierungsebene}|]|
\end{itemize}
\begin{lstlisting}
\newtheorem{def}{Definition}[section]
\newtheorem{thm}{Satz}[section]
\newtheorem*{lemma}{Lemma}
\end{lstlisting}
\newtheorem{thm}{Satz}[section]
\begin{LTXexample}
\begin{thm}[Jalea et al.]
  Sei \(a=b\) und \(b=c\). Dann ist \(a=c\).
\end{thm}
\begin{proof}
  Trivial.
\end{proof}
\end{LTXexample}
\end{frame}

%%%%%%%%%%%%%%%%%%%%%%%%%%%%%%%%%%%%%%%%%%%%%%%

\section{Physik}
\teil{Physik}
\subsection{SI-Einheiten}
\begin{frame}[fragile]{Setzen von Einheiten}
Paket \pkg{siunitx} (Joseph Wright)
\begin{LTXexample}[preset={\obeylines},pos=r]
\SI[separate-uncertainty]{23.448(5)e23}{g.cm^3}
\si[per-mode=fraction]{\joule\per\eV}
\si{\joule\per\eV}
\num[round-precision=2]{4.4583 x 3.2 e21}
\num[mode=text]{4.58}
\num[exponent-product=\cdot]{1e10}
\ang[]{45}
\end{LTXexample}
\end{frame}

\begin{frame}[fragile]{Setzen von Einheiten}
Ändern der Voreinstellungen mittels |\sisetup|
\begin{LTXexample}
\sisetup{negative-color=red}
\(\num{-3}, \num{3},
\num[negative-color=blue]{-5x5},
\num{2}\cdot\num 2\)\\

\def\a{5.1}
\(\SI{\a x 5.3}{\milli\meter}\)\\
\(\num{\a x 5.3}\,\si{\square\milli\meter}\)\\
\(\num{\a x 5.3}\,\si{\milli\meter
\squared}\)
\end{LTXexample}
\end{frame}

\begin{frame}[fragile]{Gradangaben}
\begin{LTXexample}
\ang{10}
\ang{12.3}
\ang{4,5}
\\ Heidelberg:
\ang{49;25;}N, \ang{8;43;}O,
\end{LTXexample}
\end{frame}

\begin{frame}[fragile]{Einheiten}
\begin{LTXexample}[preset=\large]
\SI{5.54}{ms^{-2}}\\
\SI{5.54}{m s^{-2}}\\
\SI{5.54}{m.s^{-2}}\\
\SI{5.54}{\meter\per \second\squared}\\
\SI{5.54}{\meter\per \square\second}\\
\end{LTXexample}
\end{frame}

\begin{frame}[fragile]{Einheiten}
\begin{LTXexample}[width=.4\textwidth]
\sisetup{per-mode=fraction}
\SI{1.23}{\joule\per\mole\per\kelvin}
\\ \sisetup{per-mode=symbol}
\SI{1.23}{\joule\per\mole\per\kelvin}
\\ \sisetup{per-mode=fraction,fraction-function=\sfrac}
\SI{1.23}{\joule\per\mole\per\kelvin}
\end{LTXexample}
\end{frame}


\subsection{Mehr Vektoren}
\begin{frame}[fragile]{Mehr Vektoren}
\begin{itemize}
\item manchmal hat man spezielle Anforderungen an die Vektorpfeile
\item Paket \pkg{esvect} bietet Anpassungen der Pfeilform
\item korrekter Satz bei Subskripten wird beachtet
\end{itemize}
\begin{LTXexample}[preset={\obeylines}]
$\vv a$
$\vec a$
\end{LTXexample}
\begin{itemize}
\item Pfeiltyp über Paketoption |[a]| bis |[h]| einstellbar
\item mögliche Pfeile: siehe Dokumentation
\end{itemize}
\end{frame}

\begin{frame}[fragile]{Mehr Vektoren}{Subskripte}
\begin{itemize}
\item Sternversion |\vv*{}{}| sorgt für passende Subskripte:
\end{itemize}
\begin{LTXexample}[preset={\obeylines}]
$\vec{ab}_{\Delta}$\\[-2ex]
$\vv {{ab}_{\Delta}}$\\[-2ex]
$\vv*{ab}{\Delta}$
\end{LTXexample}
\end{frame}


\subsection{Feynman-Graphen}
\iffalse
\section{Feynman-Graphen}
\begin{frame}[fragile]{Feynman-Graphen}
\begin{itemize}
\item verschiedene Möglichkeiten für Feynman-Graphen:
\item Paket |feynmf|
\item Paket |feyn|
\item Graphiksoftware
\item Metafont
\item TikZ/PS-Tricks
\item …
\end{itemize}
\end{frame}

\begin{frame}[fragile]{feyn}
\begin{itemize}
\item kleines, leicht bedienbares Paket
\item bietet eine Matheschrift, mit der Feynman-Graphen gesetzt werden können
\item (halb)intuitive Bedienung
\item |\feyn|: Mathemodus
\item |\Feyn|: Textmodus
\item |\Diagram|: Komplexe Diagramme
\end{itemize}
\end{frame}

\begin{frame}[fragile]{feyn}
\begin{LTXexample}
\[\feyn{f+g}\]
\[\feyn{fA} \feyn{gV}\]
\end{LTXexample}
\begin{LTXexample}
\[\Diagram{\vertexlabel^a \\
  fd \\
    & g\vertexlabel_{\mu,c} \\
  \vertexlabel^b fu\\
}
= ig\gamma_\mu (T^c)_{ab}\]
\end{LTXexample}
\pause
⇒ siehe (sehr gute) Dokumentation von |feyn|
\end{frame}
\fi


\subsection{Quantenmechanik}
\begin{frame}[fragile]{bra ket}
\begin{itemize}
\item abstrakte Darstellung von Zuständen in der Quantenmechanik
\item Unabhängigkeit von Koordinaten
\item Ket: $\langle a\rvert$, Bra: $\lvert b \rangle$
\item Skalarprodukt: Bra(c)ket: $\langle a \vert b\rangle$
\item Matrixelement: $\langle a\vert \hat O \vert b \rangle$
\end{itemize}
\end{frame}

\begin{frame}[fragile]{Satz von bra und ket}
Paket \pkg{braket}
\begin{lstlisting}
\bra a \ket b
\braket{a|\frac A B|a}
\Braket{a|\frac A B|a}
\end{lstlisting}
\end{frame}

\begin{frame}[fragile]{Akzente}
Für Operatoren benötigt man das „Dach“:
\begin{LTXexample}[width=.4\textwidth]
\(\hat A \hat{\mathrm{A}} \check a \\
\bar h \hbar \\ 
\dot a \ddot a \dddot a \ddddot a\\
\underbrace{E = mc^2}_\text{nach Einstein}\overbrace{\int_\infty}^{\text{Hinweis}}\)
\end{LTXexample}
\end{frame}

\begin{frame}[fragile]{Pfeile}
Für Spinzustände oft verwendete Notation mittels Pfeilen:
\begin{LTXexample}[width=.4\textwidth]
$\uparrow \downarrow \Uparrow \Downarrow
\Rightarrow \leftrightarrow\\
\longrightarrow \mapsto \to \rightarrow
\leftharpoondown \rightharpoonup \rightleftharpoons
\Rsh$
\end{LTXexample}
\end{frame}

\MakeShortVerb|
\begin{frame}[fragile]{mehr Pfeile}
Über- und Unterschreibungen von Pfeilen\\ (Beschriftung von Reaktionsgleichungen etc.)
\begin{LTXexample}
$\xleftarrow[unten]{oben}
 \xrightarrow[unten]{}$
\end{LTXexample}
\begin{LTXexample}
$\overleftarrow a
\overleftrightarrow b
\stackrel\leftrightarrow T$
\end{LTXexample}
\end{frame}


\subsection{Plotten in \LaTeX}
%TODO: verweis auf Diagramme am 27.11.
\begin{frame}[fragile]{Plotten in \LaTeX}
\begin{itemize}
	\item $\exists$ \pkg{gnuplottex}
	\item \pkg{PGFplots} ist besser → eigene Vorlesung
\end{itemize}
\end{frame}

\begin{frame}{gnuplot}{was ist das?}
\begin{itemize}
\item kommandozeilenorientiertes Plotprogramm
\item klein, schnell
\item unintuitive Bedienung
\item optimal für Ausführung aus Skripten
\item[⇒] passt zur Arbeitsweise mit \TeX
\item nützlich für schnelle Testplots
\item auch professionelle Qualität möglich
\end{itemize}
\end{frame}

\begin{frame}{gnuplot}{Plotten in \LaTeX}
\begin{itemize}
\item {Vorteile}: Plotbefehle direkt im Dokument\\%
  Schriften von \LaTeX\ verwaltet ⇒  passend!
\item {Nachteile}: Portabilität leidet\\%
  Plot wird bei jedem Durchlauf neu erstellt\\%
  umständlich unter Windows
\end{itemize}
\end{frame}

\begin{frame}[fragile]{gnuplot}{Verwendung}
\begin{itemize}
\item Start aus Kommandozeile (unter Windows GUI verfügbar)
\item Grundbefehl: |plot|
\item Abkürzungen aller Befehle möglich: |plot| = |pl| = |p|
\item |p sin(x)|, |p "Datensatz" using 1:3|
\item |set style data lines|, |rep|
\end{itemize}
\end{frame}

\begin{frame}[fragile]{gnuplot}{Ausgabe}
\begin{itemize}
\item gnuplot bietet riesige Vielzahl an Ausgabeformaten
\item u.\,a. ps, jpeg, mf, mp, hp500c, gif
\item direkte Anzeige: wxt (windows), X11 (Unix)
\item viele \TeX-Formate (pstex, pslatex, texdraw, eepic, emtex, \dots)
\item \emph{kein} pdf
\item aus \LaTeX: unabhängig vom Treiber
\item benötigt shell-escape um automatisiert die Plots erstellen zu können
\end{itemize}
\end{frame}

\begin{frame}[fragile]{gnuplot}{gnuplottex}
\begin{LTXexample}
\begin{gnuplot}[scale=0.4]
p sin(x)
\end{gnuplot}
\begin{gnuplot}[scale=0.4]
set style data linespoints
p "04plotdata.gpt"
\end{gnuplot}
\end{LTXexample}
\end{frame}


%%%%%%%%%%%%%%%%%%%%%%%%%%%%%%%%%%%%%%%%%%%%%%%

\section{Finetuning}
\teil{finetuning}
\subsection{Schriften}
\begin{frame}[fragile]{Matheschriften}
\begin{itemize}
\item Matheschrift muss am Anfang des Dokumentes festgesetzt werden
\item Kann nicht im Dokument geändert werden
\item Pakete freier Schriften
\item \pkg{mathpazo}
\item \pkg{cmbright}
\item \pkg{mathpazo}
\item \pkg{eulervm}
\item \pkg{libertinus}%\href{https://github.com/khaledhosny/libertinus/}{Libertinus Math}
\end{itemize}
Eine Reihe nichtfreier Schriften ist in speziellen Paketen verfügbar.
\end{frame}

\begin{frame}[fragile]{Matheschriften}
Hervorhebungen/besondere Buchstaben:
\begin{itemize}
\item Kalligraphische Buchstaben |\mathcal|
\item Serifenlose
\item Fraktur |\Re \Im|: \hfill $\Re, \Im$
\item Aufrechte Buchstaben
%\item Fettdruck (für Griechisch: Paket |\bm|)
\item „blackboard bold“ |\mathbb{R}|:\hfill $\mathbb R$
\item mit Paket \pkg{dsfont} |\matds{R}|: \hfill$\mathds{R}$
\end{itemize} 
\end{frame}

\begin{frame}[fragile]{Matheschriften}
\begin{itemize}
\item Paket \pkg{unicode-math} (Will Robertson) bietet experimentellen Zugriff auf otf-Matheschriften
\item freie Matheschriften selten
\item Unterstützung noch sehr rudimentär
\item zukünftige Entwicklung vielversprechend
\item in \LaTeX3 evtl. stabil verfügbar \dots
\item geplant für lua\TeX
\end{itemize}
\end{frame}


\subsection{Spaces}
\begin{frame}[fragile]{Änderung der Platzverteilung}
\begin{itemize}
\item Kerning
\item v/hspace: |\hspace{1cm},\hspace*{1cm}|
\item Achtung bei |\vspace|: Nur im vertikalen Modus möglich
\item Phantome
\end{itemize}
\end{frame}

\begin{frame}[fragile]{Phantome}
\begin{LTXexample}[width=.4\textwidth]
\(a_x = b\)\\
\(\hphantom{a_x} = b\)\\
\(\underline{a_x} = \underline{b\vphantom{a_x}} c \underline{a_x} \underline b\)
\end{LTXexample}
\begin{LTXexample}
\begin{align*}
a &= b\\
c &= d\\
\int a &= b
\end{align*}
\end{LTXexample}
\end{frame}

\begin{frame}[fragile]{Phantome}
\begin{LTXexample}[width=.4\textwidth]
\(a_x = b\)\\
\(\hphantom{a_x} = b\)\\
\(\underline{a_x} = \underline{b\vphantom{a_x}}\underline b\)
\end{LTXexample}
\begin{LTXexample}
\begin{align*}
a &= b\\
\vphantom{\int} c &= d\\
\int a &= b
\end{align*}
\end{LTXexample}
\end{frame}


\subsection{Smashing}
\begin{frame}{mathtools}
\begin{itemize}
\item Paket \pkg{mathtools} bietet:
\item Erweiterungen/Ergänzungen/Bugfixes zu \pkg{amsmath}
\item fine-tuning des Mathesatzes
\item Sammlung von Tricks von Michael J. Downes
\end{itemize}
\end{frame}

\begin{frame}[fragile]{mathtools}{fine-tuning: smashing}
\begin{LTXexample}[pos=t]
\[
  X = \sum_{1\le i\le j\le n} X_{ij} \qquad
  X = \sum_{\mathclap{1\le i\le j\le n}} X_{ij} \qquad
  X = \sum_{\mathclap{1\le i\le j\le n}}^{a+b+c+d} X_{ij} \qquad
  X = \smashoperator[r]{\sum_{1\le i\le j\le n}^{a+b+c+d}} X_{ij}
\]
\end{LTXexample}
\end{frame}

\begin{frame}[fragile]{mathtools}{tags}
\begin{itemize}
\item Standardform der tags ist nicht immer schön: (4)
\item Änderung mittels \pkg{amsmath}\\%
„[is] not very user friendly (it involves a macro with three @’s in its name)“
\item \pkg{mathtools}’ Weg:
\end{itemize} 
\begin{LTXexample}[width=.3\textwidth]
\newtagform{brackets}{[}{]}
\usetagform{brackets}
\begin{equation}E \neq mc^3\end{equation}
\newtagform{bfbrackets}[\textbf]{[}{]}
\usetagform{bfbrackets}
\begin{equation}E \neq mc^4\end{equation}
\end{LTXexample}
\end{frame}


\subsection{Umbrüche}
\begin{frame}{Umbruch von Formeln}
\begin{itemize}
\item nicht nur Text, sondern auch lange Formeln müssen umbrochen werden
\item sinnerhaltender Umbruch schwer
\item Umbruch nur im Inline-Mode
\item Umbruch nur bei binären Operatoren
\end{itemize}
\end{frame}

\begin{frame}[fragile]{Umbruch von Formeln}
\begin{itemize}
\item Paket \pkg{breqn} ermöglicht Umbruch in Display-Formeln
\item eigene Umgebungen: |dmath(*)| (wie |\[ \]|)
\item |dseries| 
\item |dgroup| (wie |align|)
\item |darray| (wie |eqnarray|)
\item |dsuspend| (unterbricht)
\item Befehl |\condition| für Bedingungen
\end{itemize}
\end{frame}

\begin{frame}[fragile]{Probleme}
\begin{itemize}
\item \pkg{breqn} lädt \pkg{flexisym}
\item \pkg{flexisym} definiert eigene Mathezeichen
\item[⇒] Inkompatibilität mit Schriftpaketen
\item speziell \alert{inkompatibel} zu \pkg{fontspec} (nicht mehr?)
\end{itemize}
\end{frame}


\subsection{Nummerierung}
\begin{frame}[fragile]{Nummerierung von Fallunterscheidungen}
\begin{itemize}
\item Paket \pkg{cases} bietet Nummerierung von case-Konstrukten:
\end{itemize} 
\begin{LTXexample}[pos=b]
\begin{numcases}{E = mc^2}
  m \neq 0 & Masselose Teilchen\\
  m < 0 & Antiteilchen (?)\\
  m > 0 & normale Teilchen
\end{numcases}
\end{LTXexample}
\end{frame}


\subsection{Größenänderungen}
\begin{frame}[fragile]{Relative Größenangabe}
\begin{itemize}
\item Wenn normale Schriftgrößen nicht ausreichen:\\%
|\displaystyle, \textstyle, \scriptstyle, \scripscriptstyle|
\item Paket \pkg{relsize}
\item Grundbefehle |\relsize{n}|, |n| gibt Schrittweite an
\item |\larger = \relsize{1}|
\item |\smaller = \relsize{-1}|
\item |\relscale{0.75}| – Skalierung auf den angegebenen Faktor
\item |\mathsmaller|, |\mathlarger| – Änderung der Matheschriftgröße
\end{itemize}
\end{frame}

\begin{frame}[fragile]{Relative Größenangabe}
\begin{LTXexample}[pos=b]
\[\Delta \varphi = 2
\int\limits_{r_{\min}}^{r_{\max}} \frac{ \dfrac{M}{r^2} dr} 
{\sqrt{2m (E-U) - \dfrac{M^2}{r^2}}}
\]
\end{LTXexample}
\end{frame}

\begin{frame}[fragile]{Relative Größenangabe}
\begin{LTXexample}[pos=b]
\newcommand\largeint{\mathlarger{\mathlarger{\mathlarger{\int}}}}
\[\Delta \varphi = 2
\largeint\limits_{r_{\min}}^{r_{\max}} \frac{ \dfrac{M}{r^2} dr} 
{\sqrt{2m (E-U) - \dfrac{M^2}{r^2}}}
\]
\end{LTXexample}
\end{frame}

\nocite{wikibooksmaths,amsthm, siunitx, gnuplottex, mathmode}
\begin{frame}[allowframebreaks]{Weiterführende Literatur}
\printbibliography
\end{frame}


%%%%%%%%%%%%%%%%%%%%%%%%%%%%%%%%%%%%%%%%%%%%%%%%%%%%%%%%%%%%%%%%%%%%%%%%%%%%%%%%%%%%%%%%%%%%%%

\end{document}

\section{ulem, soul}
\begin{frame}{Streichen}{durch, unter, quer, …}
\begin{itemize}
\item Pakete \pkg{ulem} und \pkg{soul} bieten verschiedene Hervorhebungsmakros\pause
\item \pkg{ulem}: \emph{underline-emphasize}
\item \pkg{soul}: \emph{space out, underline}
\end{itemize}
\end{frame}

\begin{frame}[fragile]{ulem}
\begin{itemize}
\item Hauptzweck: Ändern von |\emph| zu |\underline|
\item falls nicht gewünscht: |\normalem| oder Option |normalem|
\item andere Befehle:
\end{itemize}
\begin{LTXexample}
\uline{test}
\uuline{test}
\uwave{test}
\sout{test}
\xout{test}
\useunder{\uwave}{\bfseries}{\textbf}
\textbf{test}
\end{LTXexample}
\end{frame}

\begin{frame}[fragile]{soul}
\begin{LTXexample} 
\so{letterspacing}
\caps{CAPITALS, Small Caps}
\ul{underline}
\st{strikeout}
\hl{highlight}
\sethlcolor{blue}
\setulcolor{red}
\setulcolor{green}
\hl{highlight}
\end{LTXexample}
\end{frame}





\end{document}
