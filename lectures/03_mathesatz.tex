% !TEX program = xelatex
% !TEX encoding = UTF-8 Unicode
% !TEX spellcheck = de_DE
% 
% © 2015–2017 Moritz Brinkmann, CC-by-sa
% © 2018–2022 Maximilian Jalea, CC-by-sa
% http://latexkurs.github.io

\documentclass[
	vorläufig=false,
	datum=2022-11-09,
	titel={Mathematiksatz},
	web=false,
	max,
	aspectratio=1610,
]{../tex/latexkurs-slides}

\usepackage{
	mathtools,
	ulem,
    mathtools,
    12many,
    amsthm,
    braket,
    relsize,
%   breqn,
    cases,
    esvect,
    gnuplottex,
    soul,
    dsfont,
    siunitx,
    xfrac,
}
\normalem


\usetikzlibrary{calc,intersections,through, positioning}

\tikzset{onslide/.code args={<#1>#2}{%
  \only<#1>{\pgfkeysalso{#2}} % \pgfkeysalso doesn't change the path
}}

\begin{document}

\begin{frame}{Übersicht}
    \begin{columns}[onlytextwidth,T]
        \begin{column}{.45\textwidth}
            \tableofcontents[sections=1-2]
        \end{column}
        \begin{column}{.45\textwidth}
            \tableofcontents[sections=3-5]
        \end{column}
    \end{columns}
\end{frame}

\section{Allgmeines}
\teil{Fehlermeldungen und eigene Befehle}

\subsection{Fehlermeldungen}
\begin{frame}[t]{Umgang mit Fehlern}
\vfill
	\begin{block}{Was tun, wenn \LaTeX\ anhält?}
		\begin{itemize}
			\item Ruhe bewahren! (\texttt{tex}-Dateien können nicht beschädigt werden)
			\item Mit der Fehlersuche beim den letzten Änderungen anfangen.
			\item Ggf. Schreibfehler korrigieren.
			\item \texttt{log}-Datei Lesen!
			\item Viele Editoren helfen bei der Fehlersuche, indem sie zur Zeile springen, in der der Fehler aufgetreten ist. (Das muss nicht die fehlerhafte Zeile sein.)
		\end{itemize}
	\end{block}
\end{frame}

\begin{frame}[fragile,t]{Fehlermeldungen}
Typische Fehlermeldung:
\begin{lstlisting}
! Undefined control sequence.
l.3 Ein \Latex-Dokument
                           .
? 
! Emergency stop.
l.3 Ein \Latex-Dokument.
                           .
No pages of output.
Transcript written on document.log.
\end{lstlisting}
⇒ Befehl in Zeile 3 falsch geschrieben
\end{frame}

\begin{frame}[fragile,t]{Fehlermeldungen}
Typische Fehlermeldung:
\begin{lstlisting}
Runaway argument?
{itemize \item Erstes Item 
! Paragraph ended before \begin was complete.
<to be read again> 
                   \par 
l.60 
     
? 
\end{lstlisting}
⇒ Irgendwo nach itemize ein |}| oder ein |\end{}| vergessen.
\end{frame}

\begin{frame}{Vollständiges Minimalbeispiel}
Bei Hilfestellung in Webforen/Usenet wird in der Regel ein \emph{vollständiges Minimalbeispiel} (MWE) verlangt.
\vskip1ex
\begin{enumerate}
\item solange Code aus dem Dokument löschen, bis der Fehler gerade noch auftritt
\item alle überflüssigen Pakete entfernen
\item falls Dokumentenklasse keine Rolle spielt, \pkg{minimal} verwenden
\item wenn Fehler nur bei viel Text auftritt, \pkg{blindtext} verwenden
\end{enumerate}

\vskip1ex

Oft findet man den Fehler beim erstellen des MWE schon ganz alleine.
\end{frame}


\subsection{Eigene Befehle}
\begin{frame}[fragile]{Eigene Befehle}
\begin{itemize}
\item |\newcommand{\wasser}{H$_2$O}| ⇒ H$_2$O
\item Ermöglicht Abkürzungen im Text, die häufig vorkommen\pause
\item Änderung: |\renewcommand{\wasser}{H\kern-.1em$_2$\kern-.1em O}|:\\ H\kern-.1em$_2$\kern-.1em O
\end{itemize} 
\end{frame}

\begin{frame}[fragile,t]{Leerzeichen in \TeX}
\TeX\ „frisst“ gerne Leerzeichen – vor allem nach Befehlen:\\
|\wasser ist nass| ⇒ H$_2$Oist nass.

\pause
\begin{itemize}
\item<+-> \TeX\ liest Befehle vom |\| bis zum ersten nicht-Buchstaben%
\\ (Zahl, Klammer, Leerzeichen, Punkt, \dots)
\\ \verb*|\LaTeX   ist  manchmal   umständlich|%
\item<+-> Befehle im Text immer mit |\| oder |{}| beenden:
\item<+-> \verb*|\LaTeX\ ist manchmal umständlich.|
\end{itemize}
\begin{olcol}
\begin{block}{}\hfill{%\hspace{8em}{%
\only<2-3>{\LaTeX ist manchmal umständlich}%
\only<4>{\LaTeX\ ist manchmal umständlich}}\hfill\,
\end{block}
\end{olcol}
\end{frame}

\begin{frame}[fragile]{Befehle mit Argumenten}
\begin{lstlisting}
\newcommand\molekuel[3][H]{Das Molekül #1$_#2$#3}
\end{lstlisting}
\begin{itemize}
\item Argumente werden mit |[|\meta{Anzahl}|]| definiert
\item Optionales Argument in eckigen Klammern
\item Zugriff in der Definition möglich mit |#1|
\item In der Verwendung meist mit geschweiften Klammern |{Co}|
\end{itemize}
\newcommand\molekuel[3][H]{Das Molekül #1$_#2$#3}
|\molekuel{2}{O}| ⇒ \molekuel{2}{O}\\
|\molekuel[Co]{7}{O}| ⇒ \molekuel[Co]{7}{O}\\
\end{frame}


\section{Mathe}
\teil{Mathematische Formeln in \LaTeX}
\subsection{Modi}


\begin{frame}{Inline- vs. Display-Formeln}
\rmfamily
\textbf{Inline-Mathe:} \(E=mc^2\) kennt jedes Kind, aber kaum jemand kann wirklich mehr damit anfangen als mit \(\int^\infty_{-\infty}\sum_{n = 1}^5 dx\), wobei diese Formel nun mal gar keinen Sinn ergibt, aber zeigt, wie Grenzen im \TeX-Mathesatz aussehen.\\
\textbf{Inline-Mathe mit Displaystyle:} \(E=mc^2\) kennt jedes Kind, aber kaum jemand kann wirklich mehr damit anfangen als mit \(\displaystyle \int^\infty_{-\infty}\sum_{n = 1}^5 dx\), wobei diese Formel nun mal gar keinen Sinn ergibt, aber zeigt, wie Grenzen im \TeX-Mathesatz aussehen.\\
\textbf{Display-Mathe:} \(E=mc^2\) kennt jedes Kind, aber kaum jemand kann wirklich mehr damit anfangen als mit \[\int^\infty_{-\infty}\sum_{n = 1}^5 dx,\] wobei diese zweite Formel nun mal gar keinen Sinn ergibt, aber zeigt, wie Grenzen im \TeX-Mathesatz aussehen.
\end{frame}

\subsubsection**{Inlinemode}
\begin{frame}[fragile]{Inlinemode}
\begin{itemize}
	\item Formeln, die direkt im Fließtext vorkommen
	\item kurze Formeln, Nennung von Variablen
	\item Elemente gehen nicht über die Zeilenhöhe hinaus
	\item Grenzen werden \emph{neben} Integrale, Summen und Produkte gesetzt
\end{itemize}
\vfill
\rmfamily
\begin{LTXexample}
Seien \(m\) und \(n\) natürliche Zahlen mit \(n=5 m\).
\end{LTXexample}
\end{frame}


\begin{frame}[fragile]{Inlinemode}
Der Inlinemode ist über drei Wege zu erreichen:
\begin{itemize}
\item |\(|\meta{Formel}|\)|
\item |\begin{math}|\meta{Formel}|\end{math}|
\item |$|\meta{Formel}|$|
\end{itemize}
\pause
|\( \)| ist meist die beste Variante
\end{frame}

\begin{frame}[fragile]{Umbrüche}
Formeln können von \TeX{} umgebrochen werden:
\begin{itemize}
\item an Relationen \hfill |=|, |<|, |>|, etc.
\item an binären Operatoren \hfill |+|, |-|, etc.
\item Umbruch kann durch Gruppierung vermieden werden. \hfill |{}|
\end{itemize}
\begin{LTXexample}
Ein langer Text zum Zeilenende 
\(a + b + c\) \\
Ein langer Text bis zum Zeilenende 
\({a + b + c}\)
\end{LTXexample}
\end{frame}

\subsubsection**{Displaymode}
\begin{frame}{Displaymode}
	\begin{itemize}
		\item Auszeichnung wichtiger Formeln
		\item Darstellung langer Rechnungen
		\item komplexe Formeln
		\item mehrfach indizierte Größen
		\item geschachtelte Brüche
		\item …
	\end{itemize}
\end{frame}

\begin{frame}[fragile]{Displaymode}
\emph{klassische} Display-Formeln sind über drei Wege zu erreichen:
	\begin{itemize}
		\item |\begin{displaymath}|\meta{Formel}|\end{displaymath}|\\
		abgesetzte Formel ohne Nummerierung
		\item |\[|\meta{Formel}|\]|\\
		Abkürzung für |displaymath|
		\item |\begin{equation}|\meta{Formel}|\end{equation}|\\
		abgesetzte Formel mit Nummerierung
		\item<2> \sout{\texttt{\$\$}\meta{Formel}\texttt{\$\$}}\\\pause
		\TeX-Syntax führt in \LaTeX{} zu unerwarteten und unerwünschten Ergebnissen\\\alert{⇒ unbedingt vermeiden!}
	\end{itemize}
\end{frame}

\begin{frame}[fragile]{Display in Inline und umgekehrt}
	\begin{itemize}
		\item Dislaystyle kann mit |\displaystyle| im Inline-Modus aufgerufen werden.
	\end{itemize}
\begin{LTXexample}[pos=b]
Hier kommt ein großer Bruch, der
$\frac{a}{b} < \displaystyle \frac{a}{b}$
viel zu groß für den normalen Fließtext ist.
\end{LTXexample}
	\begin{itemize}
		\item Inlinestyle kann mit |\textstyle| im Display-Modus aufgerufen werden.
	\end{itemize}
\begin{LTXexample}[pos=b]
\[\frac 12 > \textstyle \frac 12 \]
\end{LTXexample}
\end{frame}

\begin{frame}[fragile]{Option fleqn}
	\begin{itemize}
		\item Formeln sehen oft zentriert nicht gut aus und wirken zerfleddert
		\item linksbündige Ausrichtung ggf. besser
		\item[⇒] |fleqn| als Dokumentenoption 
	\end{itemize}
\begin{lstlisting}
\documentclass[fleqn]{scrartcl}
\end{lstlisting}
\end{frame}

\subsection{\texttt{amsmath}}
\begin{frame}[fragile]{Mehrzeilige Formeln}
	Eine Reihe von untereinander ausgerichteten, zueinander angeordneten Gleichungen wird z.\,B. verwendet für:
	\begin{itemize}
		\item Herleitungen
		\item Übersichten
		\item Vergleich von Formeln
	\end{itemize}
	\pause
	\sout{\TeX-Standardumgebung: \texttt{eqnarray}} \pause \alert{unschön}\\
	\alert{besser:} |align|-Umgebung aus dem |amsmath|-Paket.
	\pause
\begin{LTXexample}
\begin{align}
a &= b, &
c &= d,\\
abc &= d \\
&= r
\end{align}
\end{LTXexample}
	ohne Nummerierung: |{align*}|
\end{frame}

\begin{frame}[fragile]{\hologo{AmS}math}
	\begin{itemize}
		\item Paket von der American Mathematical Society (\hologo{AmS})
		\item besteht aus mehreren Paketen, u.\,a.:\\%
		|amsmath|, |amssymb|, |amsfonts|%
		\item bietet umfangreiche Erweiterungen des Mathesatzes:
		\item vielfältige Umgebungen und Anpassungen
		\item neue oder verbesserte Definitionen von Befehlen
		\item Korrekturen von Abständen
		\item \only<1>{\dots}
		\pause wird mit Fehlerkorrekturen, etc. ergänzt durch |mathtools| 
		\pause
		\item[⇒] kann im Prinzip \emph{immer} geladen werden, wenn man was mit Mathe macht.
	\end{itemize}
\begin{lstlisting}
\usepackage{amsmath, mathtools}
\end{lstlisting}\pdfmarginpar{Genaugenommen reicht es aus, mathtools zu laden. Es läd amsmath automatisch.}
\end{frame}

\subsection{Grundbefehle}
\subsubsection*{Abstände}
\begin{frame}[fragile]{Abstände}
	\begin{itemize}
		\item \TeX\ bzw. \LaTeX\ bzw. geladene Pakete kontrollieren Abstände
		\item Unterschiede zwischen Variablen, Operatoren, Relationen etc.
		\item Festgelegt durch die |\mathcode|s der Zeichen
		\item Änderbar mit |\kern|, |\|, |\,| etc.
		\item \alert{niemals} Konstrukte wie |\ \ \ \ | verwenden!
		\item Besser: |\quad|, |\qquad|, |\hspace{1em}|
	\end{itemize}
\end{frame}

\subsubsection*{Größe von Formeln}
\begin{frame}[fragile]{Größenänderungen}
	\begin{itemize}
		\item Standardbefehle wie |\small|,  |\tiny|, |\Huge| haben in Formeln keine Wirkung
		\item Aber Formeln passen sich der Umgebung an
	\end{itemize} 
	\pause
\begin{LTXexample}[pos=b]
\small \[ 
  x_{n+1} = x_n - \frac{f(x_n)}{f^{\,\prime}(x_n)} 
\]
\huge \[
  x_{n+1} = x_n - \frac{f(x_n)}{f^{\,\prime}(x_n)}
\]
\end{LTXexample}
\end{frame}

\subsection{Variablen}
\begin{frame}[fragile]{Variablen und Zahlen}
	\begin{itemize}
		\item Variablen werden kursiv gesetzt: |\(a\)|: \(a\)
		\item Schriftart abhängig von der Dokumentenklasse!\\%
		(Groteske, Serifen etc.)
		\item Ziffern werden automatisch korrekt gesetzt: \(12.2\) statt 12.2
	\end{itemize}
\end{frame}

\begin{frame}[fragile]{Dezimaltrennzeichen}
im amerikanischen Satz:
\begin{LTXexample}[preset=\Large]
\(1,234.567\)
\end{LTXexample}\pause
im deutschen Satz:
\begin{LTXexample}[preset=\Large]
\(1.234,567\)
\end{LTXexample}
\alert{⇒ falsche Spationierung!}
\end{frame}

\begin{frame}[fragile]{Dezimaltrennzeichen}
\begin{block}{Einmalige Anpassung:}
	|\(1\mathpunct{.}234\mathpunct{.}567{,}89\)|\\
	\(1\mathpunct{.}234\mathpunct{.}567{,}89\) (angepasst)\\
	\(1.234.567,89\) \normalsize (nicht angepasst)\\
\end{block}
%\pause
%\begin{block}{Korrektur des Dezimaltrennzeichens}
%|\DeclareMathSymbol{,}{\mathpunct}{letters}{"3B}|\\
%|\DeclareMathSymbol{.}{\mathord}{letters}{"3A}|
%\end{block}
\pause
\begin{block}{Automatische Anpassung}
Paket |icomma| passt Dezimaltrennzeichen automatisch dokumentenweit an.\\
Andere Möglichkeit: Paket |siunitx| → siehe Vorlesung Mathesatz II
\end{block}
\end{frame}

\begin{frame}[fragile]{Hoch- und Tiefstellung}
	\begin{itemize}
		\item Zeichen mit besonderer Bedeutung: |^| und |_|
		\item Hochstellung: |a^b|\hfill \(a^b\)
		\item Tiefstellung: |a_b|\hfill \(a_b\)
		\item Gruppierungen sind möglich: |a^{bc}|, |a_{bc}|\hfill \(a_{bc}\)
		\item Kombination ist möglich: |a_b^c|\hfill \(a_b^c\)
		\item Ohne vorhergehendes Zeichen: |^{235}U|\hfill \(^{235}\mathrm U\)
		\item Schachtelung nur mit Gruppierung:\\%
		|a_{b_{c_{d_{e_{f^g}}}}}^{h^{i^{j_k}}}| \hfill \Large \(a_{b_{c_{d_{e_{f^g}}}}}^{h^{i^{j_k}}}\)\normalsize
		\item[] |a_b_c| produziert Fehler!
	\end{itemize}
\end{frame}

\subsubsection*{Operatoren}
\begin{frame}[fragile]{Operatoren}
Operatorennamen werden aufrecht gesetzt und sind vordefiniert
	\begin{itemize}
		\item richtig: \(\sin(x)\)\quad falsch: \(sin(x)\)
	\end{itemize}
\begin{LTXexample}[pos=b]
\(\sin(x) \cos(y) \tan(2\pi) \lim \arctan\)
\end{LTXexample}
	\pause
	\begin{itemize}
		\item Paket amsopn bietet viele Vordefinitionen:
	\end{itemize}
\begin{lstlisting}
\arccos \arcsin \arg \cos \cot \coth \deg \det
\exp \gcd \inf \injlim \lg \lim \limsup \ln
\max \min \projlim \sec \sinh \sup \tanh
\end{lstlisting}
\end{frame}

\begin{frame}[fragile]{Operatoren}
Sollten die vorgegebenen Definitionen nicht genügen:
\begin{lstlisting}
\usepackage{amsopn}
\DeclareMathOperator{\Res}{Res}
\end{lstlisting}
in der Präambel.
\end{frame}

\begin{frame}[fragile]{Klammern}
Klammerung von großen Ausdrücken kann Probleme bereiten:
\begin{LTXexample}
\[ (
  \frac{\int^a x dx}{\sum_{n=1} x}
) \]
\end{LTXexample}
Besser:
\begin{LTXexample}
\[ \left(
  \frac{\int^a x dx}{\sum_{n=1} x}
\right) \]
\end{LTXexample}
\end{frame}


\begin{frame}[fragile]{Klammern}
\begin{itemize}
\item |\left| und |\right| vor allem, was dehnbar ist
\item |\left( \right]| funktioniert auch
\item |\left. \right)| liefert angepasste rechte Klammer
\item Hoch- und Tiefstellung werden angepasst:
\end{itemize}
\begin{LTXexample}[pos=b]
\begin{displaymath}
  \left. \int_a^b f(x) \mathrm dx \right\vert_a^b
  \qquad
  \left\{ \int_a^b f(x) \mathrm dx \right]
\end{displaymath}
\end{LTXexample}
\end{frame}

\begin{frame}[fragile]{Grenzen}
\begin{itemize}
\item Grenzen per |\limits| angeben
\item Mehrzeilige Grenzen mit |\atop|
\item Auch Allgemein für alle Grenzen zu setzen als Option von \AmS math\\
        |\usepackage[intlimits,sumlimits]{amsmath}|
\end{itemize}
\vspace{1em}
\begin{LTXexample}
\[
  \int_a^b
  \int\limits_a^b
  \sum_{n=1}^\infty
  \prod_{n = 1 \atop m = 2}
\]
\end{LTXexample}
\end{frame}

\begin{frame}[fragile]{Sonderzeichen}
\begin{itemize}
\item Viele Zeichen sind über ihren Namen ereichbar,
\item genauso Griechische Groß- und Kleinbuchstaben
\end{itemize}
\begin{LTXexample}[preset=\vspace{-1em}]
\begin{align*}
  \nabla \square \\ 
  \partial \infty \\
  \pm \mp \\
  \alpha \beta \gamma \\
  \rho \varrho \\
  \kappa \varkappa \\
  \epsilon \varepsilon \\
  \theta \vartheta \\
   A B \Gamma 
\end{align*}
\end{LTXexample}
\pause
Wenn man ein Symbol sucht:\\
\texttt{texdoc \href{http://mirrors.ctan.org/info/symbols/math/maths-symbols.pdf}{maths-symbols} \href{http://mirrors.ctan.org/info/symbols/comprehensive/symbols-a4.pdf}{symbols-a4}}
oder \alert{\href{http://detexify.kirelabs.org/classify.html}{Detexify}}
\end{frame}

\begin{frame}[fragile]{Wurzeln}
\begin{LTXexample}[preset=\Large]
\[
  \sqrt{a_{n_{m_p}}}
  \quad
  \sqrt[3]{a}\quad
\]
\end{LTXexample}
\pause
\begin{itemize}
\item zu tiefe Unterlängen sind unschön \item[⇒] |\smash[|\meta{t, b}|]{|\meta{Formel}|}|
\end{itemize}
\begin{LTXexample}[preset=\Large]
\[
  \sqrt{a_{n_{m_p}}}
  \quad
  \sqrt{
    \smash[b]{
      a_{n_{m_p}}
    }
  }
\]
\end{LTXexample}
\end{frame}

\subsection{Vektoren, Matrizen, Tensoren}
\begin{frame}[fragile,t]{Vektoren}
Vektoren sind vielfältig darstellbar:
\begin{itemize}
\item Fettgedruckt  mit |\boldsymbol| oder |\mathbf|
\item „falscher“ Fettdruck: |\pmb|
\item Mit Pfeil drüber als |\vec|\\[-1.28em]
\only<2>{
\hspace*{-.76ex}\vspace*{-1.65ex}

$\left.
\begin{matrix}
	\text{\quad}\\%[-.5ex]
	\text{\hspace{12em}}
\end{matrix}
\right\rbrace\text{\parbox{10em}{\raggedright Typografisch unschön, nur für Handschriften}}$
\vspace*{-2.9ex}
}
\item Unterstrichen mit |\underbar|
\end{itemize}
\vfill
\begin{LTXexample}[pos=b, preset=\centering]
\( \boldsymbol a\ \mathbf a \) \\
\( \pmb a\ a \) \\
\( \vec a\ \underbar a \)
\end{LTXexample}
\end{frame}

\begin{frame}[fragile]{Matrizen}
\begin{LTXexample}
\[
  \begin{matrix}
    a_{11} & a_{12}\\
    a_{21} & a_{22}
  \end{matrix}
\]
\end{LTXexample}
\pause
\begin{LTXexample}
\[
  \left(
    \begin{matrix}
      a_{11} & a_{12}\\
      a_{21} & a_{22}
    \end{matrix}
  \right)
\]
\end{LTXexample}
\end{frame}


\begin{frame}[fragile]{Matrizen}
\AmS math definiert weitere Matrixumgebungen:\\[2em]
\begin{minipage}{3cm}
\[\begin{pmatrix}
a & b \\ c & d
\end{pmatrix}\]
\centering pmatrix
\end{minipage}
\begin{minipage}{3cm}
\[\begin{Vmatrix}
a & b \\ c & d
\end{Vmatrix}\]
\centering Vmatrix
\end{minipage}
\begin{minipage}{3cm}
\[\begin{vmatrix}
a & b \\ c & d
\end{vmatrix}\]
\centering vmatrix
\end{minipage}
\\[2em]
\begin{minipage}{3cm}
\[\begin{Bmatrix}
a & b \\ c & d
\end{Bmatrix}\]
\centering Bmatrix
\end{minipage}
\begin{minipage}{3cm}
\[\begin{bmatrix}
a & b \\ c & d
\end{bmatrix}\]
\centering bmatrix
\end{minipage}
\begin{minipage}{3cm}
\[\begin{smallmatrix}
a & b \\ c & d
\end{smallmatrix}\]\\
\centering smallmatrix
\end{minipage}
\end{frame}

\subsection{Relationen}
\begin{frame}[fragile]{Relationen}
\begin{LTXexample}
\(= \equiv \approx \asymp \bowtie \cong \dashv \doteq \sim \simeq \propto \smile\)
\end{LTXexample}
\pause Negierung mit |\not|

\begin{LTXexample} 
\(\not = \neq \not\equiv
\not \approx \not A \not\kern-.25em A
\not\kern-.2em\int \not\kern-.2em\partial \not \smile\)
\end{LTXexample}
\pause Stapeln von Symbolen

\begin{LTXexample}
\(\stackrel{oben}{unten}\)
\(\stackrel{\text e}{\text a}=\)ä
\(\stackrel . = \neq \doteq\)
\(\stackrel != \stackrel ?=\)
\end{LTXexample}
\end{frame}


\subsection{Mengen}
\begin{frame}[fragile]{12many}
\begin{itemize}
\item Paket \pkg{12many} bietet Vereinfachung und Anpassung zum Mengensatz:\\%
\(\{1, \dots, m\}\)
\item Befehle:\\
|\nto{n}{m}|, |\ito{m}|, |\oto{m}|
\item Stil ändern mit |\setOTMstyle[]{|\meta{style}|}|
\end{itemize}
\begin{LTXexample}[width=.6\textwidth]
\( \nto{i}{k},
  \ito{m},
  \oto{\alpha_i} \)
\end{LTXexample}
\end{frame}


\subsection{Integrale}
\begin{frame}[fragile]{Integrale}
\AmS{}math bietet weitere Integrale:
\begin{LTXexample}[width=.45\textwidth]
\[ \iint \iiint \iiiint \]
\[ \oint \idotsint \]
\[ \int \int \]
\end{LTXexample}
\end{frame}

\begin{frame}[fragile]{Integrale}
\begin{columns}
\begin{column}{.4\textwidth}
Zusätzliche Integraldarstellungen bieten:
\begin{itemize}
\item \pkg{wasysym}
\item \pkg{txfonts}
\item \pkg{esint}
\item \pkg{MnSymbol}
\item \pkg{mathdesign}
\end{itemize}
\end{column}
\begin{column}{.5\textwidth}
\alert{Auf Kompatibilität achten}\\
Verschiedene Matheschriften zusammen können Probleme bereiten.
\end{column}
\end{columns}
\end{frame}


\subsection{Komplexe Matrizen}
\begin{frame}[fragile]{Satz komplexer Matrizen}
\begin{LTXexample}[width=.4\textwidth]
\[
\begin{pmatrix}
  a & b     & \dots & z\\
  b & \dots & \dots & z\\
  \vdots & \ddots& \reflectbox{\(\ddots\)} 
                              & \vdots\\
  \hdotsfor{4}\\
  z & b & \dots &
      \begin{pmatrix}
        a & b \\ c & d
      \end{pmatrix}
\end{pmatrix}
\]

\end{LTXexample}
\end{frame}


\subsection{Typische Mathe-Umgebungen}
\begin{frame}[fragile]{Typische Mathe-Umgebungen}%{Definition, Satz, Beweis, Lemma, …}
Mit dem \AmS-Paket \pkg{amsthm} lassen sich typische Mathe-Umgebungen wie „Satz“ und „Beweis“ erstellen:
\begin{itemize}
\item Anlegen einer Umgebungen mit |\newtheorem{|\meta{Kürzel}|}{|\meta{Name}|}[|\meta{Nummerierungsebene}|]|
\end{itemize}
\begin{lstlisting}
\newtheorem{def}{Definition}[section]
\newtheorem{thm}{Satz}[section]
\newtheorem*{lemma}{Lemma}
\end{lstlisting}
\newtheorem{thm}{Satz}[section]
\begin{LTXexample}
\begin{thm}[Jalea et al.]
  Sei \(a=b\) und \(b=c\). Dann ist \(a=c\).
\end{thm}
\begin{proof}
  Trivial.
\end{proof}
\end{LTXexample}
\end{frame}

%%%%%%%%%%%%%%%%%%%%%%%%%%%%%%%%%%%%%%%%%%%%%%%

\section{Physik}
\teil{Physik}
\subsection{SI-Einheiten}
\begin{frame}[fragile]{Setzen von Einheiten}
Paket \pkg{siunitx} (Joseph Wright)
\begin{LTXexample}[preset={\obeylines},pos=r]
\SI[separate-uncertainty]{23.448(5)e23}{g.cm^3}
\si[per-mode=fraction]{\joule\per\eV}
\si{\joule\per\eV}
\numproduct[round-precision=2]{4.4583 x 3.2 e21}
\num[mode=text]{4.58}
\num[exponent-product=\cdot]{1e10}
\ang[]{45}
\end{LTXexample}
\end{frame}

\begin{frame}[fragile]{Setzen von Einheiten}
Ändern der Voreinstellungen mittels |\sisetup|
\begin{LTXexample}
\sisetup{negative-color=red}
\(\num{-3}, \num{3},
\numproduct[negative-color=blue]{-5x5},
\num{2}\cdot\num 2\)\\

\def\a{5.1}
\(\SI{\a}{\milli\meter}\)\\
\(\numproduct{\a x 5.3}\,\si{\square\milli\meter}\)\\
\(\numproduct{\a x 5.3}\,\si{\milli\meter
\squared}\)
\end{LTXexample}
\end{frame}

\begin{frame}[fragile]{Gradangaben}
\begin{LTXexample}
\ang{10}
\ang{12.3}
\ang{4,5}
\\ Heidelberg:
\ang{49;25;}N, \ang{8;43;}O,
\end{LTXexample}
\end{frame}

\begin{frame}[fragile]{Einheiten}
\begin{LTXexample}[preset=\large]
\SI{5.54}{ms^{-2}}\\
\SI{5.54}{m s^{-2}}\\
\SI{5.54}{m.s^{-2}}\\
\SI{5.54}{\meter\per \second\squared}\\
\SI{5.54}{\meter\per \square\second}\\
\end{LTXexample}
\end{frame}

\begin{frame}[fragile]{Einheiten}
\begin{LTXexample}[width=.4\textwidth]
\sisetup{per-mode=fraction}
\SI{1.23}{\joule\per\mole\per\kelvin}
\\ \sisetup{per-mode=symbol}
\SI{1.23}{\joule\per\mole\per\kelvin}
\\ \sisetup{per-mode=fraction,fraction-function=\sfrac}
\SI{1.23}{\joule\per\mole\per\kelvin}
\end{LTXexample}
\end{frame}


\subsection{Mehr Vektoren}
\begin{frame}[fragile]{Mehr Vektoren}
\begin{itemize}
\item manchmal hat man spezielle Anforderungen an die Vektorpfeile
\item Paket \pkg{esvect} bietet Anpassungen der Pfeilform
\item korrekter Satz bei Subskripten wird beachtet
\end{itemize}
\begin{LTXexample}[preset={\obeylines}]
$\vv a$
$\vec a$
\end{LTXexample}
\begin{itemize}
\item Pfeiltyp über Paketoption |[a]| bis |[h]| einstellbar
\item mögliche Pfeile: siehe Dokumentation
\end{itemize}
\end{frame}

\begin{frame}[fragile]{Mehr Vektoren}{Subskripte}
\begin{itemize}
\item Sternversion |\vv*{}{}| sorgt für passende Subskripte:
\end{itemize}
\begin{LTXexample}[preset={\obeylines}]
$\vec{ab}_{\Delta}$\\[-2ex]
$\vv {{ab}_{\Delta}}$\\[-2ex]
$\vv*{ab}{\Delta}$
\end{LTXexample}
\end{frame}


\subsection{Feynman-Graphen}
\iffalse
\section{Feynman-Graphen}
\begin{frame}[fragile]{Feynman-Graphen}
\begin{itemize}
\item verschiedene Möglichkeiten für Feynman-Graphen:
\item Paket |feynmf|
\item Paket |feyn|
\item Graphiksoftware
\item Metafont
\item TikZ/PS-Tricks
\item …
\end{itemize}
\end{frame}

\begin{frame}[fragile]{feyn}
\begin{itemize}
\item kleines, leicht bedienbares Paket
\item bietet eine Matheschrift, mit der Feynman-Graphen gesetzt werden können
\item (halb)intuitive Bedienung
\item |\feyn|: Mathemodus
\item |\Feyn|: Textmodus
\item |\Diagram|: Komplexe Diagramme
\end{itemize}
\end{frame}

\begin{frame}[fragile]{feyn}
\begin{LTXexample}
\[\feyn{f+g}\]
\[\feyn{fA} \feyn{gV}\]
\end{LTXexample}
\begin{LTXexample}
\[\Diagram{\vertexlabel^a \\
  fd \\
    & g\vertexlabel_{\mu,c} \\
  \vertexlabel^b fu\\
}
= ig\gamma_\mu (T^c)_{ab}\]
\end{LTXexample}
\pause
⇒ siehe (sehr gute) Dokumentation von |feyn|
\end{frame}
\fi


\subsection{Quantenmechanik}
\begin{frame}[fragile]{bra ket}
\begin{itemize}
\item abstrakte Darstellung von Zuständen in der Quantenmechanik
\item Unabhängigkeit von Koordinaten
\item Ket: $\langle a\rvert$, Bra: $\lvert b \rangle$
\item Skalarprodukt: Bra(c)ket: $\langle a \vert b\rangle$
\item Matrixelement: $\langle a\vert \hat O \vert b \rangle$
\end{itemize}
\end{frame}

\begin{frame}[fragile]{Satz von bra und ket}
Paket \pkg{braket}
\begin{lstlisting}
\bra a \ket b
\braket{a|\frac A B|a}
\Braket{a|\frac A B|a}
\end{lstlisting}
\end{frame}

\begin{frame}[fragile]{Akzente}
Für Operatoren benötigt man das „Dach“:
\begin{LTXexample}[width=.4\textwidth]
\(\hat A \hat{\mathrm{A}} \check a \\
\bar h \hbar \\ 
\dot a \ddot a \dddot a \ddddot a\\
\underbrace{E = mc^2}_\text{nach Einstein}\overbrace{\int_\infty}^{\text{Hinweis}}\)
\end{LTXexample}
\end{frame}

\begin{frame}[fragile]{Pfeile}
Für Spinzustände oft verwendete Notation mittels Pfeilen:
\begin{LTXexample}[width=.4\textwidth]
$\uparrow \downarrow \Uparrow \Downarrow
\Rightarrow \leftrightarrow\\
\longrightarrow \mapsto \to \rightarrow
\leftharpoondown \rightharpoonup \rightleftharpoons
\Rsh$
\end{LTXexample}
\end{frame}

\MakeShortVerb|
\begin{frame}[fragile]{mehr Pfeile}
Über- und Unterschreibungen von Pfeilen\\ (Beschriftung von Reaktionsgleichungen etc.)
\begin{LTXexample}
$\xleftarrow[unten]{oben}
 \xrightarrow[unten]{}$
\end{LTXexample}
\begin{LTXexample}
$\overleftarrow a
\overleftrightarrow b
\stackrel\leftrightarrow T$
\end{LTXexample}
\end{frame}


\subsection{Plotten in \LaTeX}
%TODO: verweis auf Diagramme am 27.11.
\begin{frame}[fragile]{Plotten in \LaTeX}
\begin{itemize}
	\item $\exists$ \pkg{gnuplottex}
	\item \pkg{PGFplots} ist besser → eigene Vorlesung
\end{itemize}
\end{frame}

\begin{frame}{gnuplot}{was ist das?}
\begin{itemize}
\item kommandozeilenorientiertes Plotprogramm
\item klein, schnell
\item unintuitive Bedienung
\item optimal für Ausführung aus Skripten
\item[⇒] passt zur Arbeitsweise mit \TeX
\item nützlich für schnelle Testplots
\item auch professionelle Qualität möglich
\end{itemize}
\end{frame}

\begin{frame}{gnuplot}{Plotten in \LaTeX}
\begin{itemize}
\item {Vorteile}: Plotbefehle direkt im Dokument\\%
  Schriften von \LaTeX\ verwaltet ⇒  passend!
\item {Nachteile}: Portabilität leidet\\%
  Plot wird bei jedem Durchlauf neu erstellt\\%
  umständlich unter Windows\\%
  benötigt shell-escape um automatisiert die Plots erstellen zu können
\end{itemize}
\end{frame}

\begin{frame}[fragile]{gnuplot}{Verwendung}
\begin{itemize}
\item Start aus Kommandozeile (unter Windows GUI verfügbar)
\item Grundbefehl: |plot|
\item Abkürzungen aller Befehle möglich: |plot| = |pl| = |p|
\item |p sin(x)|, |p "Datensatz" using 1:3|
\item |set style data lines|, |rep|
\end{itemize}
\end{frame}

\begin{frame}[fragile]{gnuplot}{Ausgabe}
\begin{itemize}
\item gnuplot bietet riesige Vielzahl an Ausgabeformaten
\item u.\,a. ps, jpeg, mf, mp, hp500c, gif
\item direkte Anzeige: wxt (windows), X11 (Unix)
\item viele \TeX-Formate (pstex, pslatex, texdraw, eepic, emtex, \dots)
\item \emph{kein} pdf
\item aus \LaTeX: unabhängig vom Treiber
\end{itemize}
\end{frame}

\begin{frame}[fragile]{gnuplot}{gnuplottex}
\begin{LTXexample}
\begin{gnuplot}[terminal=cairolatex,scale=0.4]
p sin(x)
\end{gnuplot}
\begin{gnuplot}[terminal=cairolatex,scale=0.4]
set style data linespoints
p "03_plotdata.gpt"
\end{gnuplot}
\end{LTXexample}
\end{frame}


%%%%%%%%%%%%%%%%%%%%%%%%%%%%%%%%%%%%%%%%%%%%%%%

\section{Finetuning}
\teil{finetuning}
\subsection{Schriften}
\begin{frame}[fragile]{Matheschriften}
\begin{itemize}
\item Matheschrift muss am Anfang des Dokumentes festgesetzt werden
\item Kann nicht im Dokument geändert werden
\item Pakete freier Schriften
\item \pkg{mathpazo}
\item \pkg{cmbright}
\item \pkg{mathpazo}
\item \pkg{eulervm}
\item \pkg{libertinus}%\href{https://github.com/khaledhosny/libertinus/}{Libertinus Math}
\end{itemize}
Eine Reihe nichtfreier Schriften ist in speziellen Paketen verfügbar.
\end{frame}

\begin{frame}[fragile]{Matheschriften}
Hervorhebungen/besondere Buchstaben:
\begin{itemize}
\item Kalligraphische Buchstaben |\mathcal| \hfill $\mathcal{A}, \mathcal{Z}$
\item Serifenlose
\item Fraktur |\Re \Im|: \hfill $\Re, \Im$
\item Aufrechte Buchstaben
%\item Fettdruck (für Griechisch: Paket |\bm|)
\item „blackboard bold“ |\mathbb{R}|:\hfill $\mathbb R$
\item mit Paket \pkg{dsfont} |\matds{R}|: \hfill$\mathds{R}$
\end{itemize} 
\end{frame}

\begin{frame}[fragile]{Matheschriften}
\begin{itemize}
\item Paket \pkg{unicode-math} (Will Robertson) bietet experimentellen Zugriff auf otf-Matheschriften
\item freie Matheschriften selten
\item Unterstützung noch sehr rudimentär
\item zukünftige Entwicklung vielversprechend
\item in \LaTeX3 evtl. stabil verfügbar \dots
\item geplant für lua\TeX \, (aktuell einfache Konvertierung in runtime)
\end{itemize}
\end{frame}


\subsection{Spaces}
\begin{frame}[fragile]{Änderung der Platzverteilung}
\begin{itemize}
\item Kerning
\item v/hspace: |\hspace{1cm},\hspace*{1cm}|
\item Achtung bei |\vspace|: Nur im vertikalen Modus möglich
\item Phantome
\end{itemize}
\end{frame}

\begin{frame}[fragile]{Phantome}
\begin{LTXexample}[width=.4\textwidth]
\(a_x = b\)\\
\(\hphantom{a_x} = b\)\\
\(\underline{a_x} = \underline{b\vphantom{a_x}} c \underline{a_x} \underline b\)
\end{LTXexample}
\begin{LTXexample}
\begin{align*}
a &= b\\
c &= d\\
\int a &= b
\end{align*}
\end{LTXexample}
\end{frame}

\begin{frame}[fragile]{Phantome}
\begin{LTXexample}[width=.4\textwidth]
\(a_x = b\)\\
\(\hphantom{a_x} = b\)\\
\(\underline{a_x} = \underline{b\vphantom{a_x}}\underline b\)
\end{LTXexample}
\begin{LTXexample}
\begin{align*}
a &= b\\
\vphantom{\int} c &= d\\
\int a &= b
\end{align*}
\end{LTXexample}
\end{frame}


\subsection{Smashing}
\begin{frame}{mathtools}
\vfill
Paket \pkg{mathtools} bietet:
\begin{itemize}
\item Erweiterungen/Ergänzungen/Bugfixes zu \pkg{amsmath}
\item fine-tuning des Mathesatzes
\item Sammlung von Tricks von Michael J. Downes
\end{itemize}
\end{frame}

\begin{frame}[fragile]{mathtools}{fine-tuning: smashing}
\begin{LTXexample}[pos=t]
\[
  X = \sum_{1\le i\le j\le n} X_{ij} \qquad
  X = \sum_{\mathclap{1\le i\le j\le n}} X_{ij} \qquad
  X = \sum_{\mathclap{1\le i\le j\le n}}^{a+b+c+d} X_{ij} \qquad
  X = \smashoperator[r]{\sum_{1\le i\le j\le n}^{a+b+c+d}} X_{ij}
\]
\end{LTXexample}
\end{frame}

\begin{frame}[fragile]{mathtools}{tags}
\begin{itemize}
\item Standardform der tags ist nicht immer schön: (4)
\item Änderung mittels \pkg{amsmath}\\%
„[is] not very user friendly (it involves a macro with three @’s in its name)“
\item \pkg{mathtools}’ Weg:
\end{itemize} 
\begin{LTXexample}[width=.3\textwidth]
\newtagform{brackets}{[}{]}
\usetagform{brackets}
\begin{equation}E \neq mc^3\end{equation}
\newtagform{bfbrackets}[\textbf]{[}{]}
\usetagform{bfbrackets}
\begin{equation}E \neq mc^4\end{equation}
\end{LTXexample}
\end{frame}


\subsection{Umbrüche}
\begin{frame}{Umbruch von Formeln}
\begin{itemize}
\item nicht nur Text, sondern auch lange Formeln müssen umbrochen werden
\item sinnerhaltender Umbruch schwer
\item Umbruch nur im Inline-Mode
\item Umbruch nur bei binären Operatoren
\end{itemize}
\end{frame}

\begin{frame}[fragile]{Umbruch von Formeln}
\begin{itemize}
\item Paket \pkg{breqn} ermöglicht Umbruch in Display-Formeln
\item eigene Umgebungen: |dmath(*)| (wie |\[ \]|)
\item |dseries| 
\item |dgroup| (wie |align|)
\item |darray| (wie |eqnarray|)
\item |dsuspend| (unterbricht)
\item Befehl |\condition| für Bedingungen
\end{itemize}
\end{frame}

\begin{frame}[fragile]{Probleme}
\begin{itemize}
\item \pkg{breqn} lädt \pkg{flexisym}
\item \pkg{flexisym} definiert eigene Mathezeichen
\item[⇒] Inkompatibilität mit Schriftpaketen
\item speziell \alert{inkompatibel} zu \pkg{fontspec} (nicht mehr?)
\end{itemize}
\end{frame}


\subsection{Nummerierung}
\begin{frame}[fragile]{Nummerierung von Fallunterscheidungen}
\begin{itemize}
\item Paket \pkg{cases} bietet Nummerierung von case-Konstrukten:
\end{itemize} 
\begin{LTXexample}[pos=b]
\begin{numcases}{E = mc^2}
  m \neq 0 & Masselose Teilchen\\
  m < 0 & Antiteilchen (?)\\
  m > 0 & normale Teilchen
\end{numcases}
\end{LTXexample}
\end{frame}


\subsection{Größenänderungen}
\begin{frame}[fragile]{Relative Größenangabe}
\begin{itemize}
\item Wenn normale Schriftgrößen nicht ausreichen:\\%
|\displaystyle, \textstyle, \scriptstyle, \scripscriptstyle|
\item Paket \pkg{relsize}
\item Grundbefehle |\relsize{n}|, |n| gibt Schrittweite an
\item |\larger = \relsize{1}|
\item |\smaller = \relsize{-1}|
\item |\relscale{0.75}| – Skalierung auf den angegebenen Faktor
\item |\mathsmaller|, |\mathlarger| – Änderung der Matheschriftgröße
\end{itemize}
\end{frame}

\begin{frame}[fragile]{Relative Größenangabe}
\begin{LTXexample}[pos=b]
\[\Delta \varphi = 2
\int\limits_{r_{\min}}^{r_{\max}} \frac{ \dfrac{M}{r^2} dr} 
{\sqrt{2m (E-U) - \dfrac{M^2}{r^2}}}
\]
\end{LTXexample}
\end{frame}

\begin{frame}[fragile]{Relative Größenangabe}
\begin{LTXexample}[pos=b]
\newcommand\largeint{\mathlarger{\mathlarger{\mathlarger{\int}}}}
\[\Delta \varphi = 2
\largeint\limits_{r_{\min}}^{r_{\max}} \frac{ \dfrac{M}{r^2} dr} 
{\sqrt{2m (E-U) - \dfrac{M^2}{r^2}}}
\]
\end{LTXexample}
\end{frame}


\nocite{mathmode, dante:mathe, amsmath, wikibooksmaths,amsthm, siunitx, gnuplottex, mathmode}
\begin{frame}[allowframebreaks]{Weiterführende Literatur}
\printbibliography
\end{frame}

% Was noch fehlt: Definition von Kommandos
% Normen
% Skalarprodukte

\end{document}
