% !TEX program = xelatex
% !TEX encoding = UTF-8 Unicode
% !TEX spellcheck = de_DE
% 
% © 2015–2017 Moritz Brinkmann, CC-by-sa
% http://latexkurs.github.io

\documentclass[
	vorläufig=false,
	datum=2019-01-28,
	titel={Eigene Klassen und Pakete schreiben},
	web=true,
%	noshortverb=true,
	max,
	aspectratio=1610,
]{../tex/latexkurs-slides}



\begin{document}

\begin{frame}{Übersicht}
	\tableofcontents
\end{frame}


\section{Paket schreiben}
\subsection{Identifizierung}
\begin{frame}[fragile]{Begrüßung}
|\NeedsTeXFormat{|\meta{Format}|}[|\meta{Datum}|]|\\
|\ProvidesPackage{|\meta{Name}|}[|\meta{Datum}| |\meta{weitere Infos}|]|\\
|\ProvidesClass{|\meta{Name}|}[|\meta{Datum}| |\meta{weitere Infos}|]|
\begin{lstlisting}
\NeedsTeXFormat{LaTeX2e}[1996/12/01]
\ProvidesPackage{meinpaket}[2015/02/05 v0.1 Dolles Paket]
\end{lstlisting}

\end{frame}

\subsection{Laden von Paketen}
\begin{frame}[fragile]{Pakete Laden}
|\RequirePackage[|\meta{Paketoptionen}|]{|\meta{Paket}|}[|\meta{Datum}|]|\\
|\LoadClass[|\meta{Klassenoptionen}|]{|\meta{Klasse}|}[|\meta{Datum}|]|

\vfill
\pause

\begin{block}{Eigene Optionen an geladenes Paket weitergeben:}
|\RequirePackageWithOptions{|\meta{Paket}|}[|\meta{Datum}|]|\\
|\LoadClassWithOptions{|\meta{Klasse}|}[|\meta{Datum}|]|\\[1ex]
\end{block}

\vfill
Eine Klasse kann nur (einmal) von einer Klasse geladen werden
\vfill

\begin{lstlisting}
\RequirePackage[hmargin=3cm]{geometry}
\end{lstlisting}

\end{frame}

\subsection[fragile]{Paketoptionen}
\begin{frame}[fragile]{Optionen}
\begin{itemize}
\item Option definieren:\\
|\DeclareOption{|\meta{Option}|}{|\meta{Code}|}|
\item nicht definierte Optionen verwenden:\\
|\DeclareOption*{|\meta{Code}|}|
\item Optionen verarbeiten:\\
|\ProcessOptions|
\item innerhalb von |\DeclareOprtion*|:\\
|\CurrentOption|\\
|\OptionNotUsed|
\end{itemize}
\vfill

\begin{lstlisting}
\DeclareOption{a4paper}{%
  \setlength{\paperheight}{297mm}%
  \setlength{\paperwidth}{210mm}%
}
\DeclareOption*{\OptionNotUsed}
\ProcessOptions
\end{lstlisting}
\end{frame}

\begin{frame}[fragile]{|key=value|-Optionen}
Klassen/Paketoptionen mit Key-Value-Syntax lassen sich zum Beispiel mit \pkg{kvoptions} realisieren.
\vfill	
\begin{lstlisting}
\SetupKeyvalOptions{
  family=meinpaket,
  prefix=mypkg@
}
\DeclareStringOption[default]{mystring}
\DeclareBoolOption{mybool}
\ProcessKeyvalOptions{mypkg}
\end{lstlisting}
\end{frame}


\subsection{Befehle}
\begin{frame}[fragile]{Makros Definieren}
\begin{itemize}
\item Befehl definieren:\\
|\newcommand{|\meta{Befehl}|}[|\meta{Anzahl}|][|\meta{Default}|]{|\meta{Definition}|}|
\item Befehl umdefinieren:\\
|\renewcommand{|\meta{Befehl}|}[|\meta{Anzahl}|][|\meta{Default}|]{|\meta{Definition}|}|
\item Befehl nur definieren, falls er nicht existiert:\\
|\providecommand{|\meta{Befehl}|}[|\meta{Anzahl}|][|\meta{Default}|]{|\meta{Definition}|}|
\item Testen ob ein Befehl (genau so) definiert ist:\\
|\CheckCommand{|\meta{Befehl}|}[|\meta{Anzahl}|][|\meta{Default}|]{|\meta{Definition}|}|
\end{itemize}
\vfill\pause
Oder mit \hologo{LaTeX3}-Syntax (\pkg{expl3}, siehe letzte Vorlesung)
\end{frame}

\begin{frame}[fragile]{Nützliche Befehle}
\begin{block}{Code zu verschiedenen Zeitpunkten ausführen}
|\AtBeginDocument{|\meta{Code}|}|\\
|\AtEndDocument{|\meta{Code}|}|\\[1ex]
|\AtEndOfPackage{|\meta{Code}|}|\\
|\AtEndOfClass{|\meta{Code}|}|
\end{block}
\end{frame}

\subsection{Errors, Warnings, Infos}
\begin{frame}[fragile]{Mit dem Nutzer sprechen}
|\typeout{|\meta{Nachricht}|}|\\
|\PackageInfo{|\meta{Paket}|}{|\meta{Nachricht}|}|\\
|\PackageWarning{|\meta{Paket}|}{|\meta{Nachricht}|}|\\
|\PackageWarningNoLine{|\meta{Paket}|}{|\meta{Nachricht}|}|\\
|\PackageError{|\meta{Paket}|}{|\meta{Nachricht}|}{|\meta{Hilfetext}|}|
\vfill

\begin{lstlisting}
\PackageInfo{meinpaket}{Dies ist eine Info.}
\PackageWarning{meinpaket}{Dies ist eine Warnung.}
\PackageError{meinpaket}{Dies ist ein Fehler.}{Fehler lässt sich nicht beheben.}
\end{lstlisting}
\end{frame}


\section{Paket benutzen}
\subsection{\TeX-Directory-Structure}
\begin{frame}[fragile]{Paket einbinden}
Im Dokument: |\usepackage{meinpaket}|

|meinpaket.sty| muss im selben Ordner liegen
\vfill\pause

Alternative: \TeX\ durchsucht alle Ordner des TDS-Baums\\
Lokale Pakete können in |$TEXMFHOME| abgelegt werdern

\end{frame}



\section{Paket verpacken}
\subsection{Doc und DocStrip}
\begin{frame}[fragile]{Pakete ausliefern}

Programm \pkg{DocStrip} kann aus einer Datei verschiedene Ausgabe-Dokumente erstellen.

\begin{enumerate}
\item Lösche alle Zeilen, die mit |%| anfangen \hfill → |sty| oder |cls|
\item Lösche alle |%| die am Anfang der Zeile stehen \hfill → |pdf|
\end{enumerate}
\vfill
\overleaf{tex1400}
\end{frame}


\begin{frame}[fragile,t]{Beispiel dtx\quad\small 1/4}
\begin{lstlisting}
% \iffalse meta-comment
% Copyright (C) 2015 by Lieschen Müller
% \fi \iffalse
%<driver>\ProvidesFile{meinpaket.dtx}
%<package>\NeedsTeXFormat{LaTeX2e}[2007/07/20]
%<package>\ProvidesPackage{meinpaket}[2015/02/05 v0.1 Dolles Paket]
%<*batchfile>
\begingroup
\input{docstrip.tex}
\preamble
Copyright (C) 2015 by Lieschen Müller
\endpreamble
\askforoverwritefalse
\generate{\file{meinpaket.sty}{\from{meinpaket.dtx}{package}}}
\endgroup
%</batchfile>
\end{lstlisting}
\end{frame}

\begin{frame}[fragile,t,]{Beispiel dtx\quad\small 2/4}
\begin{lstlisting}
%<*driver>
\documentclass{ltxdoc}
\usepackage[ngerman,english]{babel}
\usepackage[T1]{fontenc}
\usepackage[utf8]{inputenc}
\begin{document}
\DocInput{meinpaket.dtx}
\end{document}
%</driver>
% \fi
% \CheckSum{0}
% 
% \changes{v0.1}{2015/02/05}{Initial version}
% 
% \GetFileInfo{meinpaket.dtx}
% 
\end{lstlisting}
\end{frame}

\begin{frame}[fragile,t]{Beispiel dtx\quad\small 3/4}
\begin{lstlisting}
% \title{Mein Paket\thanks{Diese Anleitung bezieht sich auf Version \fileversion}}
% \author{Lieschen Müller}
% \date{\filedate}
% \maketitle
% 
% \begin{abstract}
%   \noindent Dieses tolle Paket tut tolle Dinge.
% \end{abstract}
% 
% \tableofcontents
% 
% \section{Anleitung}
% So funktioniert mein tolles Paket ...
% 
\end{lstlisting}
\end{frame}


\begin{frame}[fragile,t]{Beispiel dtx\quad\small 4/4}
\begin{lstlisting}
% \section{Implementierung}
% So habe ich mein Paket implementiert:
% 
% \iffalse
%<*package> 
% \fi
%    \begin{macrocode}
\providecommand{\meinbefehl}{Hier steht der eigentliche Inhalt des Pakets}
%    \end{macrocode}
% \iffalse 
%</package>
% \fi
% 
\endinput
\end{lstlisting}
\end{frame}



\nocite{clsguide,dtxtut,docstrip}
\begin{frame}{Weiterführende Literatur}
\printbibliography
\end{frame}

\end{document}
