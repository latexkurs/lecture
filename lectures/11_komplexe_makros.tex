% !TEX program = lualatex
% !TEX encoding = UTF-8 Unicode
% !TEX spellcheck = de_DE
% 
% © 2015–2017 Moritz Brinkmann, CC-by-sa
% http://latexkurs.github.io

\documentclass[
	vorläufig=false,
	datum=2022-01-26,
	titel={Komplexe Makros und Befehle},
	web=true,
%	noshortverb=true,
	max,
	aspectratio=1610,
]{../tex/latexkurs-slides}

\begin{document}

\begin{frame}{Übersicht}
	\tableofcontents
\end{frame}


\section{Verschiedenes}
\frame{\centering\alert{Teil I}\\\huge Verschiedenes}
\subsection{Poster}
\begin{frame}{Poster}
∃ diverse Klassen für Satz von (wissenschaftlichen) Postern: \pkg{a0poster}, \pkg{sciposter}, \pkg{tikzposter}\\[1ex]\pause

Empfehlung: \pkg{tikzposter}\\[1ex]

Nutzt	 \TikZ\ um Objekte (Blocks, etc.) auf Poster zu platzieren. Bedienung vergleichar mit \pkg{beamer}.
\end{frame}

\subsection{Vorlesungsmitschriften}

\begin{frame}{Mitschreiben}
\begin{itemize}
	\item in Vorlesungen oder Übungen mit\TeX{}en manchmal nützlich
	\item entweder extrem hohe Tippgeschwindigkeit nötig
	\item oder durchdachte Befehlsdefinitionen
	\item \emph{wichtig:} alle strukturelle Information muss vorhanden sein!\\%
(auch, wenn es nicht gut aussieht)
\end{itemize} 
\end{frame}

\begin{frame}[fragile]{Mitschreiben}
\begin{itemize}
	\item häufig nur stichpunktartiges Aufschreiben
	\item[⇒] |\obeylines|
	\item Aufzählungen abkürzen, z.\,B. mittels |\let\+\item|
	\item …
\end{itemize}
\end{frame}


\section{Makros in \LaTeXe}
\frame{\centering\alert{Teil II}\\\huge \LaTeXe}
\subsection{newcommand, newenvironment \& Co}
\begin{frame}[fragile]{Makros in \LaTeX}
Zur Definition eigener Befehle in \LaTeX\ verfügbar:\\
|\newcommand|, |\renewcommand|, |\newenvironment|\\[2ex]
\pause
|\|(|re|)|newcommand{|\meta{Befehlsname}|}|\\
|  [|\meta{Anzahl der Argumente}|]|\\
|  [|\meta{Default für erstes (optionales) Argument}|]|\\
|  {|\meta{Befehlsdefinition}|}|\\[2ex]
|\newenvironment{|\meta{Umgebungsname}|}|\\
|  [|\meta{Anzahl der Argumente}|]|\\
|  [|\meta{Default für erstes (optionales) Argument}|]|\\
|  {|\meta{Definition Code \emph{vor} Umgebung}|}|\\
|  {|\meta{Definition Code \emph{nach} Umgebung}|}|\\[2ex]

\pause
Varianten mit Stern: |\newcommand*| für zusätzliche Fehler-Checks, falls Argumente \emph{keine} Umbrüche/Leerzeilen enthalten dürfen
\end{frame}

\subsection{def und let}
\begin{frame}[fragile]{Makros in \TeX}
\TeX\ bietet die Primitiven |\def| und |\let|\\[2ex]\pause

|\def|\meta{Befehlsname}\meta{Argument(e)}|{|\meta{Befehlsdefinition}|}|\\
\begin{lstlisting}
\def\mymakro#1#2{Makro mit zwei Argumenten #1 und #2}
\end{lstlisting}\pause
\vspace*{2ex}

|\let|\meta{neuer Befehlsname}\meta{alter Befehlsname}\\
\begin{lstlisting}
\let\newmakro\oldmakro
\end{lstlisting}
\begin{itemize}
\item generiert |\newmakro| mit exakt den selben Eigenschaften wie |\oldmakro|
\item wenn sich |\oldmakro| ändert, bleibt |\newmakro| erhalten
\end{itemize}
\end{frame}


\begin{frame}[fragile]{Makros in \TeX}
\begin{itemize}
\item |\def| und |\let| auch in \LaTeX\ verfügbar
\item High-Level Befehle wie |\newcommand| sind meist vorzuziehen
\item |\let| manchmal praktisch
\item  nur benutzen, wenn man weiß was man tut
\end{itemize}
\end{frame}


\subsection{Naming Conventions}
\begin{frame}[fragile]{Naming Conventions}
\begin{description}
\item[|lowercase|] Endnutzer-Befehle auf Dokumenten-Level\\(braucht man ständig)
\item[|MixedCase|] Befehle für spezielle Funktionen in Paketen oder Klassen\\(braucht man selten)
\item[|with@sign|] interne Befehle in Paketen oder im \LaTeX-Kernel\\(braucht man \emph{nie})
\end{description}
\vfill\pause
\begin{block}{spezieller Schutzmechanismus}
@-Zeichen hat anderen \textenglish{category code} als normale Buchstaben, Befehle mit @ werden daher ignoriert.\\
Ausschalten: |\makeatletter|\\
wieder Einschalten: |\makeatother|
\end{block}
\vfill\pause
\TeX-Primitiven sind – aus historischen Gründen – auch lowercase
\end{frame}


\section{\LaTeX3}
\frame{\centering\alert{Teil III}\\\huge \LaTeX3}
\begin{frame}{Makros in \LaTeX3}
	\begin{columns}\hspace{-2.5em}
		\begin{column}{.7\textwidth}
			\begin{itemize}
				\item Mit \LaTeX3 wird alles besser:
				\begin{itemize}
					\item Konsequente Unterscheidung zwischen Nutzer-, Design- und Programmierebene
					\item Namespaces für Pakete
					\item sehr bequeme und flexible Befehlsdefinitionen
				\end{itemize}
				\item \LaTeX3-Syntax schon jetzt nutzbar:
				\begin{itemize}
					\item Paket \pkg{expl3} für Entwickler
					\item Paket \pkg{xparse} für Endnutzer
				\end{itemize}
			\end{itemize}
		\end{column}
		\begin{column}{.2\textwidth}
			\vspace{-2cm}
			\includegraphics[width=1.5\textwidth]{11_expl3}
		\end{column}
	\end{columns}
\end{frame}


\subsection{Makros in \LaTeX3}
\begin{frame}[fragile]{Makros in \LaTeX3}%{xparse}

Mit Paket \pkg{xparse} verfügbar:\\
|\NewDocumentCommand|, |\RenewDocumentCommand|, |\NewDocumentEnvironment|, …
\vfill

|\NewDocumentCommand{|\meta{Befehlsname}|}|\\|  {|\meta{Argumentstruktur}|}|\\|  {|\meta{Definition}|}|
\vfill

 \meta{Argumentstruktur} beschreibt wie viele und welche Argumente\\der Befehl annimmt (sozusagen die Signatur)
\end{frame}


\begin{frame}[fragile]{Argumentstruktur}
\begin{block}<1>{mandatorische Argumente}
\begin{description}
\item[|m|] klassisches mandatorisches Argument \hfill |{|\meta{...}|}|
\item[|l|] liest alles vor der nächsten Klammer \hfill \meta{...}|{|
\item[|r|\meta{t1}\meta{t2}] alles zwischen \meta{t1} und \meta{t2}\quad
	z.\,B. |r<>| \hfill |<|\meta{...}|>|
%\item[|R|\meta{token1} und \meta{token2}|{|\meta{Default}|}|] wie |r| nur mit default-Wert
\item[|u\{|\meta{t}|\}|] liest alles bis \meta{t}\quad
	z.\,B. |u{§&}| \hfill \meta{...}|§&|
\item[|v|] Verbatim-Input \hfill \texttt{\textbar\meta{...}\textbar}\\
	Eingabe wird nicht interpretiert \hfill \texttt{\{\meta{...}\}}\\
\end{description}
\end{block}
\vfill
\begin{lstlisting}
\NewDocumentCommand{\mycommand}{ m l m r^° }
  {(#1,#2,#3,#4)}
\mycommand{eins}zwei{drei}^vier°
\end{lstlisting}
\end{frame}

\begin{frame}[fragile]{Argumentstruktur}
\begin{block}{optionale Argumente}
\begin{description}
\item[|o|] klassisches optionales Argument \hfill |[|\meta{...}|]|
\item[|O\{|\meta{df}|\}|] wie |o| mit Default-Wert \quad z.\,B. |O{default}| \hfill |[|\meta{...}|]|
\item[|d|\meta{t1}\meta{t2}] alles zwischen \meta{t1} und \meta{t2}\quad
	z.\,B. |d:+| \hfill |:|\meta{...}|+|
\item[|D|\meta{t1}\meta{t2}|\{|\meta{df}|\}|] wie |d| mit Default-Wert \quad
	z.\,B. |d(){bla}| \hfill |(|\meta{...}|)|
\item[|s|] Stern, setzt |\BooleanTrue|, falls vorhanden \hfill |*|
\end{description}
\end{block}
\vfill
\begin{lstlisting}
\NewDocumentCommand{\mycommand}
  { d<| O{zwei} s D|>{vier} }
  { (#1,#2,#4) \IfBooleanT{#3}{:-)} }
\mycommand<eins|[2]*
\end{lstlisting}
\end{frame}

\begin{frame}[fragile]{Argumentstruktur}
\begin{block}{Argument-Modifier}
\begin{description}
\item[|+|\meta{Arg-Kürzel}] erlaubt Eingabe von Umbrüchen innerhalb eines Arguments \\
	z.\,B. |+m|
\item[|>\{|\meta{Prozessor}|\}|] Argumente vor dem Auslesen bearbeiten \\
	z.\,B. |> {\ReverseBoolean} m|\\|> {\TrimSpaces} o|
\end{description}
\end{block}
\hfill
\begin{lstlisting}
\NewDocumentCommand{\mycommand}
  { >{\ReverseBoolean} s o +m }
  { \IfBooleanTF{#1}{kein stern}{stern} #2 #3 }
\mycommand*{Text mit\\Umbruch}
\end{lstlisting}
\end{frame}

\begin{frame}[fragile]{Umgebungen}
|\NewDocumentEnvironment{|\meta{Umgebungsname}|}|\\
|  {|\meta{Argumentstruktur}|}|\\
|  {|\meta{Definition Code \emph{vor} Umgebung}|}|\\
|  {|\meta{Definition Code \emph{nach} Umgebung}|}|\vfill

\begin{lstlisting}
\newDocumentEnvironment{myquote}{ o }
  {\begin{quote}\sffamily\itshape}
  {\end{quote}\IfNoValueF{#1}{Quelle:#1}}
  
\begin{myquote}[Internet]
  Bla bla, Chemtrails, Lügenpresse …
\end{myquote}
\end{lstlisting}
\end{frame}


\begin{frame}[fragile]{expl3}
Erweiterte \LaTeX3-Funktionalität für Entwickler mit \pkg{expl3} verfügbar\\[1ex]

|\ExplSyntaxOn|\\
|  |\meta{Code}\\
|\ExplSyntaxOff|\\[1ex]

Schaltet neue Syntax ein und aus
\end{frame}

\section{\hologo{LuaLaTeX}}
\frame{\centering\alert{Teil IV}\\\huge \hologo{LuaLaTeX}}
\begin{frame}[fragile]{Nutzung von Lua mit \hologo{LuaLaTeX}}
Innerhalb von \TeX\ Lua-Code eingeben:\\
|\directlua{|\meta{Lua-Code}|}|\\[1ex]

Innerhalb von Lua-Code etwas an \TeX\ ausgeben:\\
|tex.print(|\meta{TeX-Ausgabe}|)|\\[3ex]
\pause

\begin{LTXexample}
$\pi = \directlua{
  tex.sprint(math.pi)
}$
\end{LTXexample}

\end{frame}

\begin{frame}[fragile]{Nutzung von Lua mit \hologo{LuaLaTeX}}
\begin{itemize}
	\item |\directlua| macht manchmal Probleme
	\begin{itemize}
		\item bei Umbrüchen
		\item bei Lua-Kommentaren |---|
		\item bei Sonderzeichen |_^&${}%|
	\end{itemize}
	\item Paket \pkg{luacode} behebt viele dieser Probleme.:\\[1ex]
|\begin{luacode*}|\\
|  |\meta{Lua-Code}\\
|\end{luacode*}|

	\item in Variante mit Stern sind keine \TeX-Befehle möglich
	\item in Variante ohne Stern werden \TeX-Makros expandiert
\end{itemize}

\end{frame}

\nocite{xparse, interface3, expl3, lualatex-doc-de, luacode}
\begin{frame}[shrink]{Weiterführende Literatur}
\printbibliography
\end{frame}

\end{document}
