% !TEX program = xelatex
% !TEX encoding = UTF-8 Unicode
% !TEX spellcheck = de_DE
% 
% © 2015–2017 Moritz Brinkmann, CC-by-sa
% © 2018–2022 Maximilian Jalea, CC-by-sa
% http://latexkurs.github.io

\documentclass[
	vorläufig=false,
	datum=2022-12-14,
	titel={Präsentationen mit beamer},
	web=true,
	aspectratio=1610,
%	noshortverb=true,
	max,
]{../tex/latexkurs-slides}


\usepackage{
	siunitx,
}

\begin{document}

\begin{frame}{Übersicht}
	\tableofcontents
\end{frame}


\begin{frame}{Vorbemerkungen}
	\begin{itemize}
		\item \LaTeX\ ist \emph{nicht} für Präsentationen gedacht
		\item spezielle Programme häufig besser geeignet
		\item Wahl des Programms vom Inhalt abhängig
	\end{itemize}
\end{frame}

\begin{frame}{Präsentationen in \LaTeX}
	Standardklasse |slides| für die Erstellung von (Overhead-)Folien\pause
	\vfill
	
	\LaTeX\ Bietet eine Menge spezialisierter Klassen und Pakete zum Satz von Präsentationen:
	\begin{itemize}
		\item \pkg{beamer} % größte und wichtigste klasse
		\item \pkg{powerdot} % weiteste verbreitung zusammen mit beamer, allerdings für latex (DVI)
		\item \pkg{prosper} % frühe Klasse, die den speziellen Präprozessor verwendet
		\item \pkg{lecturer} % Einfache Präsentationen, freie positionierung von Objekten
		\item \pkg{elpres} 
		\item …
	\end{itemize}
\end{frame}



\section{Präsentationen mit beamer}
\frame{\centering\alert{Teil I}\\\huge Präsentationen mit \texttt{beamer}}

\begin{frame}{Präsentationen mit beamer}
	\begin{itemize}
		\item Erstellen von bildschirmfüllenden „Folien“
		\item ansprechende Farbgebung
		\item strukturierte Darstellung des Inhaltes
		\item dynamische Effekte
		\item multimediale Unterstützung
	\end{itemize}
\end{frame}

\begin{frame}{Präsentationen mit beamer}
	\begin{block}{Das beamer-Prinzip}
		Seitengröße wird auf \SI{128}{mm}\,×\,\SI{96}{mm} gesetzt.\\So kann man mit \emph{normalen} Schriftgrößen arbeiten, die im Fullscreen-Modus riesig aussehen.
		
		⇒ automatischer Schutz vor zu vollen Folien
	\end{block}
\end{frame}

\begin{frame}[fragile]{Präsentationen mit beamer}
	\begin{itemize}
		\item alle Pakete, Befehle, Umgebungen (fast) wie normal zu verwenden
		\begin{itemize}
			\item |\tableofcontents| erzeugt Inhaltsverzeichnis
			\item |\begin{tabular}| setzt Tabelle
			\item …
		\end{itemize}
		\item spezielle Umgebung enthält den Inhalt einzelner Folien\\
		|\begin{frame}|%\\	(powerdot: |slide|)
		%\item spezielle Befehle für Overlays (Nach-und-nach-Einblenden von Elementen)
	\end{itemize}
\end{frame}

\begin{frame}[fragile]{frames}
%|\frame{|\meta{Inhalt}|}|\\[1ex]
|\begin{frame}[|\meta{Optionen}|]{|\meta{Titel}|}{|\meta{ggf. Untertitel}|}|
\vspace{1em}
	\begin{itemize}
		\item Umgebung |frame| erzeugt eine „Folie“
		\item erstes Argument: Titel, zweites: Untertitel
		\item optionales Argument |[fragile]| nötig für |\verb| u.\,ä.
		\item Jede pdf-Seite ist ein statisches Objekt
		\item[⇒] Überblendeffekte benötigen mehrere Seiten (innerhalb einer Folie)
	\end{itemize}
\end{frame}

\begin{frame}[fragile]{Ein erstes beamer-Dokument}
\begin{olcol}
\begin{lstlisting}
\documentclass{beamer}

\begin{document}

  \title{Doller Vortrag}
  \author{Hans Wurst}
  
  \frame{\titlepage}

  \begin{frame}{Erste Folie}
    Inhalt der Ersten Folie
  \end{ frame}

\end{document}
\end{lstlisting}
\end{olcol}
\end{frame}


\begin{frame}[fragile,t]{vertikale Ausrichtung}
vertikale Ausrichtung mittels optionalem Argument |[t,b,c]|, auch als Dokumentklassenoption
\vspace{2ex}

|\begin{frame}[t]{|\meta{Titel}|}{|\meta{Untertitel}|}|\\
|  |\meta{Folieninhalt}\\
|\end{frame}|
\end{frame}

\begin{frame}{Überblendeffekte}
\begin{itemize}
    \item für dynamische Effekte: |<|\meta{Kürzel}|>|
    \item<+-> |<+->|\phantom{\texttt{5}} lässt Objekt erscheinen und bleiben
    \item<+> |<+>|\phantom{\texttt{-5}} lässt Objekt erscheinen und wieder verschwinden
    \item<4> |<4>|\phantom{\texttt{-5}} Objekt erscheint auf Folie 4
    \item<4-5> |<4-5>| Objekt erscheint auf Folien 4 bis 5  
    \item<5> |<0>| Objekt erscheint gar nicht
\end{itemize}
\end{frame}
    
\begin{frame}[fragile]{Überblendeffekte}
z.\,B. bei |itemize|:

\begin{lstlisting}
\begin{itemize}[<+->]  % Angabe gilt für alle \items
  \item<+-> Punkt 1
  \item<4> Punkt 2
  \item<+-> Punkt 3
\end{itemize}
\end{lstlisting}
Auch bei |\includegraphics<|\meta{Kürzel}|>| u.\,a.
\end{frame}

\begin{frame}[fragile]{Überblendeffekte}{Pause}
\begin{itemize}
    \item {|\pause|} stoppt den Inhalt an beliebiger Stelle
    \item erste Seite wird bis |\pause| gesetzt
    \pause
    \item zweite Seite enthält den gesamten Inhalt (bis zum nächsten |\pause|)
\end{itemize}
\[a =\pause b_{c\pause \cdot d}\]
\end{frame}

\begin{frame}[fragile]{Überblendeffekte}{only}
\begin{itemize}
    \item |\only<|\meta{Kürzel}|>{|\meta{Inhalt}|}| setzt den \meta{Inhalt} nur in den angegeben Seiten
    \item Platz für den \meta{Inhalt} wird \emph{nicht} freigehalten
    \item |\only<4>{|\meta{Inhalt}|}| setzt nur in der vierten Seite
    \item |\only<3->{|\meta{Inhalt}|}| setzt ab der dritten Seite
\end{itemize}
\end{frame}

\begin{frame}[fragile]{Überblendeffekte}{uncover}
\begin{itemize}
    \item|\uncover<|\meta{Kürzel}|>{|\meta{Inhalt}|}| setzt den \meta{Inhalt} nur in den angegeben Seiten
    \item Platz für den \meta{Inahlt} \emph{wird} freigehalten
    \item |\uncover<4>{|\meta{Inhalt}|}| setzt nur in der vierten Seite
    \item |\uncover<3->{|\meta{Inhalt}|}| setzt ab der dritten Seite
\end{itemize}
\end{frame}

\begin{frame}[fragile]{Strukturelemente}{block}
\begin{LTXexample}[rframe={}]
\begin{block}{Titel}
  Inhalt eines schön gefärbten Blocks.
\end{block}
\begin{block}<2>{Zwei}
  Und noch einer.
\end{block}
\end{LTXexample}
\end{frame}

\begin{frame}[fragile]{Strukturelemente}{theorem}
\begin{LTXexample}[rframe={}]
\begin{theorem}[Trautmann et al.]
  1 + 2 = 3
\end{theorem}
\begin{proof}
  2 = 1+1\\
  1+1+1 = 3
\end{proof}
\begin{example}
  2+1 = 3
\end{example}
\end{LTXexample}
Konflikt mit |theorem| aus \pkg{amsmath}!\\
Umgebungen können nummeriert werden mit Dokumentenoption |[envcountsec]|
\end{frame}

\begin{frame}[fragile]{themes}{allgemeine}
\begin{itemize}
    \item themes sind Stilvorlagen, die das gesamte Layout beeinflussen
    \item Einbinden mittels |\usetheme| im Header
    \item benannt nach Tagungsorten
    \item siehe \pkg{beamer}-Dokumentation oder \url{http://hartwork.org/beamer-theme-matrix/}
\end{itemize}
\end{frame}

\begin{frame}[fragile]{themes}{inner}
\begin{itemize}
    \item beeinflussen das Aussehen von Elementen in der Folie
    \item Aufzählungen, Abbildungsbeschriftung, Boxen etc.
    \item |\useinnertheme|
\end{itemize} 
\end{frame}

\begin{frame}[fragile]{themes}{outer}
\begin{itemize}
    \item beeinflussen das Aussehen der äußeren Element
    \item Kopfzeile, Fußzeile, Navigation etc.
    \item |\useoutertheme|
\end{itemize} 
\end{frame}

\begin{frame}[fragile]{themes}{color}
\begin{itemize}
    \item wie der Name sagt …
    \item je nach Theme werden verschiedene Elemente coloriert
    \item Farben für jedes Element anpassbar:\\
\begin{lstlisting}
\setbeamercolor{footnote}{fg=red}
\end{lstlisting}
    \item |fg| für |foreground|, |bg| für |background|
\end{itemize}
\end{frame}

\begin{frame}[fragile]{themes}{font}
\begin{itemize}
    \item ändert Auswahl der Schriftarten
    \item |default| (Serifenlose), |serif|, |structurebold|, |structuresmallcapserif|, …\\
    |professionalfont| (für professionelle (gekaufte) Schriften)
\end{itemize}
\end{frame}

\begin{frame}[fragile]{Navigationselemente}
	\begin{center}
	\scalebox{2}{
		\insertslidenavigationsymbol \insertframenavigationsymbol \insertsubsectionnavigationsymbol \insertsectionnavigationsymbol \insertdocnavigationsymbol \insertbackfindforwardnavigationsymbol}
	\end{center}
	\begin{itemize}
		\item Erlauben Springen zwischen Folien, Frames, (Sub-)Sections, …
		\item normalerweise in der rechten unteren Ecke
		\item Ausblenden mit |\beamertemplatenavigationsymbolsempty|
	\end{itemize}
\end{frame}

\begin{frame}[fragile]{Gliederung}
\begin{itemize}
    \item normale Gliederungselemente vorhanden
    \item |\section, \subsection, \chapter, ...|
    \item Angabe von |\section| bewirkt zunächst nichts!\\%
(Absatzüberschriften werden \emph{nicht} ausgegeben)
    \item Einfluss nur in Inhaltsverzeichnissen und Headern
\end{itemize}
\end{frame}


\section{Multimedia}
\begin{frame}[fragile]{Gleitumgebungen}
\begin{itemize}
    \item Einfügen von Abbildungen, Tabellen u.\,ä. wie gewohnt
    \item Gleitumgebungen werden nicht nummeriert
    \item Positionsangaben (h,t,b) werden ignoriert
    \item |\logo| fügt ein Logo global in die Präsentation ein (z.\,B. oben links)
    \item Bilder einfügen mittels |\includegraphics| oder:
\end{itemize}
\begin{lstlisting}
\pgfdeclareimage[height=0.5cm]{logo}{tu-logo}
\logo{\pgfuseimage{logo}}
\logo{\includegraphics[height=0.5cm]{logo}{tu-logo}}
\end{lstlisting}
\end{frame}

\begin{frame}{Filme}
\begin{itemize}
    \item Paket |multimedia| (gehört zu beamer) laden
    \item unter Verwendung von pdf\LaTeX\ und geeignetem Viewer: Einbinden von Videos möglich
\end{itemize}
\end{frame}

\iftrue
\begin{frame}[fragile]{Modi}
\begin{itemize}
	\item \pkg{beamer} kann mit verschiedenen Modi umgehen
	\item |presentation| (Standard), |handout|, |article|, …
	\item |handout| entfernt alle overlays
	\item |\only<|\meta{Modus}|>{|\meta{Inhalt}|}|
\end{itemize}
\begin{lstlisting}
\begin{frame}<handout:0> %versteckt ganze Folie
  \only<4|article:3>{Bla}
  ...
\end{lstlisting}
\end{frame}
\fi

% modi: \begin{frame}<0| handout:0>; handout entfernt overlays
% shrink, plain, allowframebreak
% \beamertemplatenavigationsymbolsempty
% Literatur: beamer (besonders abschnitt 5)

%%%%%%%%%%%%%%%%%%%%%%%%%%%%%%%%%%%%%%%%
\section{PDF-Viewer}
\frame{\centering\alert{Teil II}\\\huge PDF-Viewer}

\begin{frame}{Präsentationssoftware}
	Kriterien für eine gute Präsentationssoftware
	\begin{itemize}
		\item fullscreen-Modus
		\item Bedienung mit Tastatur und Maus möglich
		\item schwärzen\,/\,weißen des Schirms
		\item schnelle Navigation zwischen Folien
		\item Implementierung aller pdf-Features
		\item Kennzeichnungen\,/\,Hervorhebungen während der Präsentation
		\item eigene Überblendmechanismen
		\item kein Blockieren des pdfs!
	\end{itemize}
\end{frame}

\begin{frame}{\TeX works}
	\begin{itemize}
		\item frei verfügbar (= offener Quellcode)
		\item hervorragender Editor mit eingebautem Viewer
		\item nötige Änderungen in der Präsentation können on-the-fly vorgenommen werden
		\item sync\TeX\ bereitet mit beamer große Probleme!
		\item nicht alle pdf-features vorhanden
		\item nicht besonders für Präsentationen geeignet
	\end{itemize}
\end{frame}

\begin{frame}{Adobe Acrobat Reader}
	\begin{itemize}
		\item kostenlose Software
		\item nicht \emph{frei} (im Sinne von Open Content)
		\item für Windows, Mac, (Linux) verfügbar
		\item implementiert sämtliche pdf-Features (z.\,B. Videos möglich)
		\item bietet einige Präsentationsfeatures (Bildschirm schwarz\,/\,weiß etc.)
		\item blockiert das pdf!
	\end{itemize}
\end{frame}

\begin{frame}{okular}
	\begin{itemize}
		\item vielfältiger Viewer
		\item implementiert (scheinbar) alle pdf-features\\%
(kann Videos abspielen, Transitions etc.)
	\end{itemize}
\end{frame}

\begin{frame}{zathura}
    \begin{itemize}
        \item sehr schlanker PDF Betrachter
        \item implementiert die Routinen (PDF-Libraries) von Poppler und MuPDF
        \item Bedienung orientiert sich an VIM
        \item mit |F5| wird der Präsentationsmodus gestartet
        \item mit |--fork| kann zathura aus der Konsole als eigene Instanz gestartet werden
        \item mit |-s| kann zathura wie auch in \TeX Studio vorhanden aus der PDF in den Quelltext springen (mit richtiger Konfiguration in z.B. Kyle)
        \item nach Rekompilierung öffnet zathura das neue PDF automatisch
    \end{itemize}
    ⇒ Im Allgemeinen mein Go-To PDF-Betrachter
\end{frame}

\begin{frame}{impressive!}
	\begin{itemize}
		\item speziell für Präsentationen erstellt
		\item freie Software (⇒ für alle Platformen verfügbar)
		\item Start aus Kommandozeile
		\item Effekte nur über Kommandozeilenargumente steuerbar!
		\item ermöglicht nützliche Präsentationseffekte: Schirm schwärzen, Spotlight, helle Rahmen ziehen, schnelle Navigation …
	\end{itemize}
\end{frame}

\begin{frame}{pdfpc – pdf-presenter-console}
	\begin{itemize}
		\item Wie impressive! nur mit mehr Augenmerk auf einfache Bedienbarkeit und weniger auf Aussehen
		\item derzeit nur für Linux verfügbar
	\end{itemize}
\end{frame}


\begin{frame}{Bonuscontent}{Wie Donald Knuth Vorträge hält …}
	 \centering
	 \qrcode{https://youtu.be/eKaI78K_rgA} \\[1em]
	\url{https://youtu.be/eKaI78K_rgA} \\
	Großartiger Vortrag von Knuth zum 32. (2\textasciicircum 5) Jubiläum von \TeX.
\end{frame}


\nocite{beamer, dante:praesentationen, presentationzen}
\begin{frame}[allowframebreaks]{Weiterführende Literatur}
\printbibliography
\end{frame}

\end{document}
