% !TEX program = pdflatex
% !TEX encoding = UTF-8 Unicode

\documentclass[totpages,german,helvetica,openbib]{europecv}

\usepackage[ngerman]{babel}
\usepackage[T1]{fontenc}

\usepackage[a4paper,top=1.27cm,left=1cm,right=1cm,bottom=2cm]{geometry}
\usepackage{bibentry}
\usepackage{hyperref}
\usepackage{graphicx}
%\ecvLeftColumnWidth{4cm}
\renewcommand{\ttdefault}{phv} % Uses Helvetica instead of fixed width font

\ecvname{Nachname(n), Vorname(n)}
%\ecvfootername{Vorname(n) Nachname(n)}
\ecvaddress{Straße, Hausnummer, Postleitzahl, Ort, Staat }
\ecvtelephone[(Falls nicht relevant, bitte löschen (siehe Anleitung)]{(Falls nicht relevant, bitte löschen (siehe Anleitung)}
\ecvfax{(Falls nicht relevant, bitte löschen (siehe Anleitung)}
\ecvemail{\url{email@address.com} (Falls nicht relevant, bitte löschen (siehe Anleitung)}
\ecvnationality{(Falls nicht relevant, bitte löschen (siehe Anleitung)}
\ecvdateofbirth{(Falls nicht relevant, bitte löschen (siehe Anleitung)}
\ecvgender{(Falls nicht relevant, bitte löschen (siehe Anleitung)}
%\ecvpicture[width=2cm]{mypicture}
\ecvfootnote{Weitere Informationen finden Sie unter \url{http://europass.cedefop.eu.int}\\
\textcopyright~ Europäische Gemeinschsften, 2003.}

\begin{document}
\selectlanguage{german}


\begin{europecv}
\ecvpersonalinfo[20pt]

\ecvitem{\large\textbf{Gewünschte Beschäftigung / 
Gewünschtes Berufsfeld}}{\large\textbf{(Falls nicht relevant, bitte löschen (siehe Anleitung)}}

\ecvsection{Berufserfahrung}
\ecvitem{Datum}{Mit der am kürzesten zurückliegenden Berufserfahrung beginnen und für jeden relevanten Arbeitsplatz separate Eintragungen vornehmen. Falls nicht relevant, Zeile bitte löschen (siehe Anleitung).}
\ecvitem{Beruf oder Funktion}{\ldots}
\ecvitem{Wichtigste Tätigkeiten und Zuständigkeiten}{\ldots}
\ecvitem{Name und Adresse des Arbeitgebers}{\ldots}
\ecvitem{Tätigkeitsbereich oder Branche}{\ldots}

\ecvsection{Schul- und berufsbildung}

\ecvitem{Datum}{Mit der am kürzesten zurückliegenden Maß nahme beginnen und für jeden abgeschlossenen Bildungs- und Ausbildungsgang separate Eintragungen vornehmen. Falls nicht relevant, Zeile bitte löschen (siehe Anleitung).}
\ecvitem{Bezeichnung der erworbenen Qualifikation}{\ldots}
\ecvitem{Hauptfächer/berufliche Fähigkeiten}{\ldots}
\ecvitem{Name und Art der Bildungs- oder Ausbildungseinrichtung}{\ldots}
\ecvitem{Stufe der nationalen oder internationalen Klassifikation}{\ldots}


\ecvsection{Persönliche Fähigkeiten und Kompetenzen}

\ecvmothertongue[5pt]{Muttersprache angeben}
\ecvitem{\large Sonstige Sprache(n)}{}
\ecvlanguageheader{(*)}
\ecvlanguage{Sprache}{}{}{}{}{}
\ecvlanguage{Sprache}{}{}{}{}{}
\ecvlanguagefooter[10pt]{(*)}

\ecvitem[10pt]{\large Soziale Fähigkeiten und Kompetenzen}{\ldots}
\ecvitem[10pt]{\large Organisatorische Fähigkeiten und Kompetenzen}{\ldots}
\ecvitem[10pt]{\large Technische Fähigkeiten und Kompetenzen}{\ldots}
\ecvitem[10pt]{\large IKT-Kenntnisse und Kompetenzen}{\ldots}
\ecvitem[10pt]{\large Künstlerische Fähigkeiten und Kompetenzen}{\ldots}
\ecvitem[10pt]{\large Sonstige Fähigkeiten und Kompetenzen}{\ldots}
\ecvitem{\large Führerschein(e)}{\ldots}

\ecvsection{Zusätzliche Angaben}
\ecvitem[10pt]{}{Hier weitere Angaben machen, die relevant sein können, z. B. zu Kontaktpersonen, Referenzen usw. Falls nicht relevant, Rubrik bitte löschen (siehe Anleitung)}
\bibliographystyle{plain}
\nobibliography{publications}
\ecvitem{}{\textbf{Publikationen}}
\ecvitem{}{\bibentry{pub1}}
\ecvitem[10pt]{}{\bibentry{pub2}}
\ecvitem{}{\textbf{rivate Interessen und Hobbies}}
\ecvitem{}{\ldots}

\ecvsection{Anlagen}
\ecvitem{}{Gegebenenfalls Anlagen auflisten. Falls nicht relevant, Rubrik bitte löschen (siehe Anleitung)}
\end{europecv}


\end{document}  
