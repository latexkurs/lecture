% !TEX program = xelatex,
% !TEX encoding = UTF-8 Unicode
% !TEX spellcheck = de_DE

\documentclass[DIV=16,twocolumn]{scrartcl}

\usepackage{fontspec}
\setmainfont{Linux Libertine O}
\newfontfamily\arabicfont[Script=Arabic]{FreeFarsi}

\usepackage{polyglossia, bidi}
\setmainlanguage{german}
\setotherlanguage{arabic}



\begin{document}
Es regierte einst in den ältesten Zeiten und verflossenen Äonen ein König von den Sassaniden auf den Inseln Indiens und Chinas, der viele Truppen und Verbündete, Diener und zahlreiches Gefolge besaß. Auch hatte er zwei wackere, tapfere Söhne, von denen jedoch der ältere noch tapferer war, als der jüngere; er herrschte über viele Länder und war so gerecht gegen seine Untertanen, daß ihn alle sehr liebten.
  
Sein Name war Scheherban, sein jüngerer Bruder hieß Schahseman, und war König von Samarkand in Persien. Beide hatten ihre Heimat nicht verlassen und jeder regierte höchst glücklich 20 Jahre lang in seinem Reiche. Da sehnte sich der ältere König nach seinem jüngeren Bruder, und befahl seinem Vezier, zu jenem hinzureisen und ihn zu ihm zu bringen. Der jüngere Bruder gehorchte alsbald und machte Anstalten zur Reise, und ließ Zelte, Kamele, Maultiere, Diener und Gefolge herbeikommen. Die Regierung war indes dem Vezier übertragen und der König reiste ab nach dem Lande seines Bruders. Um Mitternacht erinnerte er sich, etwas im Schlosse vergessen zu haben; als er dahin zurückkam, fand er seine Frau in vertrautem Umgang mit einem schwarzen Sklaven; bei diesem Anblick ward die ganze Welt schwarz in seinen Augen; er dachte, wenn dies schon vorfällt, ehe ich kaum die Stadt verlassen, was wird diese Verruchte tun, wenn ich einmal weit entfernt bin?

Er zog sein Schwert und erstach beide; dann ließ er sogleich wieder aufbrechen und reiste immer fort, bis er in die Nähe der Hauptstadt seines Bruders kam. Dort ließ er seinem Bruder durch Boten seine Ankunft melden. Dieser erschien sehr erfreut, um ihn zu begrüßen, ließ er die Stadt beleuchten, setzte sich zu ihm und unterhielt sich aufs angenehmste mit ihm. Aber der König Schahseman dachte an die Begebenheit mit seiner Gemahlin, und dieses kränkte ihn so tief, daß er bleich wurde und sein Körper an Kraft abnahm.

Als sein Bruder ihn in diesem Zustande sah, dachte er, dies ist gewiß, weil er von seinem Lande und Königreiche entfernt lebt; er ließ ihn deshalb in Ruhe und fragte nach nichts. Doch eines Tages sagte er zu ihm: »O mein Bruder! Ich sehe, dein Körper wird immer schwächer und deine Farbe bleicher.« Jener antwortete ihm: »Ich habe eine innere Krankheit«; aber er sagte ihm nicht, was er von seiner Frau gesehen. Hierauf versetzte der ältere: »Ich möchte, daß du mit mir auf die Jagd gingest, vielleicht wird dich dies zerstreuen;« da jener sich aber weigerte, ging er allein fort. Nun waren im Schlosse des jüngeren Königs, d. h. das der jüngere Bruder bewohnte, Fenster, die auf den Garten seines Bruders gingen. Hier sah er auf einmal die Türe des Schlosses sich öffnen, und zwanzig Sklavinnen und zwanzig Sklaven herauskommen; in ihrer Mitte ging die Frau seines Bruders, ausgezeichnet schön und von bewundernswertem Wuchse.

%Als sie, d.\,h. die Sklavinnen, zu einem Teiche gelangt waren, entkleideten sie sich und setzten sich zu den Sklaven. Da rief die Königin: »Masud!« und es kam ein schwarzer Sklave und umarmte sie und sie umarmte ihn. Die übrigen Sklaven taten dasselbe mit den Sklavinnen, und so brachten sie den ganzen Tag zu mit Küssen und Umarmungen. Als der Bruder des Königs dies sah, dachte er bei sich: bei Gott! mein Unglück ist geringer als dieses; dies ist mehr als mir geschehen! Kummer und Gram fühlte er nun plötzlich weichen und er konnte wieder essen und trinken.
  
%  \begin{Arabic}
  \fontspec{FreeFarsi}
    {وقد حكمت في وقت مبكر في وقت مبكر ومرت ملكا للساسانيين على جزر الهند والصين، التي كانت تمتلك العديد من القوات والحلفاء والخدمين والعديد من المتقاعدين. كان لديه أيضا اثنين من الشجعان، شجعان أبناء، منهم، ومع ذلك، كان الأكبر حتى أكثر شجاعة من الأصغر سنا. وحكم العديد من البلدان وكان صالحا جدا ضد رعاياه أن الجميع أحبه كثيرا.
    
    وكان اسمه شهربان، وكان شقيقه الأصغر شاهسمان، وكان ملك سمرقند في بلاد فارس. كلاهما لم يتركوا وطنهم وحكم كل منهم بسعادة لمدة 20 عاما في مملكته. ثم شدد الملك الأكبر على شقيقه الأصغر، وأمر وزيره بالذهاب إلى ذلك الشخص وتقديمه إليه. الشقيق الأصغر طاع على الفور، وجعل الاستعدادات للرحلة، وجلب الخيام والجمال والبغال والخدمين والوفد المرافق. الحكومة، ومع ذلك، عهد إلى الوزير، والملك غادر لبلد شقيقه. في منتصف الليل وتذكر نسيان شيء في القلعة. وعندما عاد، وجد زوجته حميمة مع عبد أسود. في هذا المنظور أصبح العالم كله أسود في عينيه. كان يعتقد، إذا حدث هذا، قبل أن أغادر المدينة، ما الذي سوف يفعله هذا الشر مرة واحدة بعيدا عني؟
    
    ووجه سيفه وطعن على حد سواء. ثم غادر فورا واستمر حتى اقترب من عاصمة أخيه. هناك أرسل كلمة وصوله إلى أخيه عبر الرسل. وكان من دواعي سروره أن يرحب به، لو كانت المدينة مضيئة وجلست وتحدثت إليه أكثر من غيره. لكن الملك شسمان فكر في هذه المناسبة مع زوجته، وهذا جرحه بعمق حتى أصبح شاحب و جسده انخفض في قوته.
    
    عندما رأى أخاه في تلك الدولة، كان يعتقد، هذا مؤكد، لأنه يعيش بعيدا عن أرضه وممالكه؛ ولذلك تركه وحده وطلب لا شيء. ولكن يوم واحد قال له: يا أخي! أرى جسمك يزداد ضعفا ولونك بالألوان ". أجاب:" لدي مرض داخلي ". لكنه لم يقل له ما رأى من زوجته. أجاب الرجل الأكبر سنا: "أود أن أذهب إلى الصيد معي، ربما هذا سوف يبدد لكم"، ولكن كما رفض، غادر وحده. الآن في قلعة الملك الأصغر، د. ح. أن الأخ الأصغر ساكن، والنوافذ التي ذهبت إلى حديقة شقيقه. هنا رأى فجأة باب القلعة مفتوحا، وعشرين عبيد وعشرين عبيد خرجوا. في خضم ذلك ذهب زوجة أخيه، جميلة رائعة ونمو مثير للإعجاب.
    
عندما، أنا. العبيد الذين جاءوا إلى بركة، خلعوا وجلسوا إلى العبيد. ثم صاحت الملكة: "مسعود!" وجاء عباد أسود واحتضنتها وعانقته. وبقية العبيد فعلوا الشيء نفسه مع العبيد، وهكذا قضوا طوال اليوم التقبيل والمعانقة. عندما رأى شقيق الملك هذا، فكر لنفسه، الله! بلدي التعاسة هو أقل من هذا؛ هذا هو أكثر من القيام به بالنسبة لي! الحزن والأسى شعر فجأة لينة وانه يمكن أن تأكل وتشرب مرة أخرى.
%  	  }
 % \end{Arabic}
\end{document}