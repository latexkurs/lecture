% !TEX program = lualatex
\documentclass{scrartcl}

\usepackage{chickenize, polyglossia}
\setmainlanguage[variant=old]{german}

%\colorstretch				% shows grey boxes that visualise the badness and font expansion line-wise
%\letterspaceadjust		% improves the greyness by using a small amount of letterspacing
%\substitutewords			% replaces words by other words (chosen by the user)
%	\addtosubstitutions{Zebra}{Giraffe}
%\variantjustification	% Justification by using glyph variants
%\suppressonecharbreaks % suppresses linebreaks after single-letter words

%\boustrophedon				% invert every second line in the style of archaic greek texts
%\countglyphs					% counts the number of glyphs in the whole document
%\countwords					% counts the number of words in the whole document
%\leetspeak						% translates the (latin-based) input into 1337 5p34k
%\medievalumlaut			% changes each umlaut to normal glyph plus “e” above it
%\randomuclc					% alternates randomly between uppercase and lowercase
%\rainbowcolor				% changes the color of letters slowly according to a rainbow
%\randomcolor					% prints every letter in a random color
%\tabularasa					% removes every glyph from the output and leaves an empty document
%\uppercasecolor			% makes every uppercase letter colored

%\chickenize					% replaces every word with “chicken” (or user-adjustable words)
%\guttenbergenize			% deletes every quote and footnotes
%\hammertime					% U can't touch this!
%\kernmanipulate			% manipulates the kerning (tbi)
%\matrixize						% replaces every glyph by its ASCII value in binary code
%\randomerror					% just throws random (La)%\TeX%\ errors at random times
%\randomfonts					% changes the font randomly between every letter
%\randomchars					% randomizes the (letters of the) whole input
%
%\BEClerize
%\dubstepenize
%\rongorongonize
%\zebranize

\begin{document}

\noindent Es war entsetzlich kalt; es schneite und der Abend dunkelte bereits; es war der letzte Abend im Jahre, Sylvesterabend. In dieser Kälte und in dieser Finsternis ging auf der Straße ein kleines armes Mädchen mit bloßem Kopfe und mit nackten Füßen. Es hatte wohl freilich Pantoffeln angehabt, als es von Hause fortging, aber das waren die seiner verstorbenen Mutter gewesen und da sie ihr nicht paßten, so hatte sie die Kleine verloren, als sie über die Straße eilte, während zwei Wagen in rasender Eile vorüberjagten; der eine Pantoffel war nicht wieder aufzufinden und mit dem andern machte sich ein Knabe aus dem Staube.

Da ging nun das kleine Mädchen auf den nackten zierlichen Füßchen, die vor Kälte ganz rot und blau waren. In ihrer alten Schürze trug sie eine Menge Schwefelhölzer und ein Bund hielt sie in der Hand. Während des ganzen Tages hatte ihr niemand etwas abgekauft, niemand ein Almosen gereicht. Hungrig und frostig schleppte sich die arme Kleine weiter und sah schon ganz verzagt und eingeschüchtert aus. Die Schneeflocken fielen auf ihr langes blondes Haar, das schön gelockt über ihren Nacken hinabfloß. Aus allen Fenstern strahlte heller Lichterglanz und über alle Straßen verbreitete sich der Geruch von köstlichem Gänsebraten. Es war ja Sylvesterabend und dieser Gedanke erfüllte alle Sinne des kleinen Mädchens.

In einem Winkel zwischen zwei Häusern, von denen das eine etwas weiter in die Straße vorsprang als das andere, kauerte es sich nieder. Seine kleinen Beinchen hatte es unter sich gezogen, aber es fror nur noch mehr und wagte es trotzdem nicht, nach Hause zu gehen, da es noch kein Schächtelchen mit Streichhölzern verkauft, noch keinen Pfennig erhalten hatte. Es hätte gewiß vom Vater Schläge bekommen, und kalt war es zu Hause ja auch; sie hatten das bloße Dach über sich und der Wind pfiff schneidend hinein, obgleich Stroh und Lumpen in die größten Ritzen gestopft waren. Ach, wie gut mußte die Wärme eines Schwefelhölzchens thun! Wenn es nur wagen dürfte, eines aus dem Schächtelchen herauszunehmen, es gegen die Wand zu streichen und die Finger daran zu wärmen! Endlich zog das Kind eines heraus. „Ritsch!“ wie sprühte es, wie brannte es. Das Schwefelholz strahlte eine warme helle Flamme aus, wie ein kleines Licht, als es das Händchen um dasselbe hielt. Es war ein merkwürdiges Licht; es kam dem Mädchen vor, als säße es vor einem großen eisernen Ofen mit Messingbeschlägen und Messingverzierungen; das Feuer brannte so schön und wärmte so wohlthuend! Die Kleine streckte schon die Füße aus, um auch diese zu wärmen — da erlosch die Flamme. Der Ofen verschwand — sie saß mit einem Stümpfchen des ausgebrannten Schwefelholzes in der Hand da.

Ein neues wurde angestrichen, es brannte, es leuchtete, und die Stelle der Mauer, auf welche der Schein fiel, wurde durchsichtig wie ein Flor. Die Kleine sah gerade in die Stube hinein, wo der Tisch gedeckt stand und köstlich dampfte die gebratene Gans darauf. Und was noch herrlicher war, die Gans sprang aus der Schüssel und watschelte mit Gabel und Messer im Rücken über den Fußboden hin; gerade auf das arme Mädchen zu. Da erlosch das Schwefelholz und nur die dicke kalte Mauer war zu sehen.

Sie zündete ein neues an. Da saß die Kleine unter dem herrlichsten Weihnachtsbaum; er war noch größer und noch weit reicher ausgeputzt als der, den sie am heiligen Abende bei dem reichen Kaufmann durch die Glasthüre gesehen hatte. Tausende von Lichtern brannten auf den grünen Zweigen, und bunte Bilder schauten auf sie hernieder; die Kleine streckte beide Hände nach ihnen in die Höhe — da erlosch das Schwefelholz. Die vielen Weihnachtslichter stiegen höher und höher und sie sah jetzt erst, daß es die hellen Sterne waren. Einer von ihnen fiel herab und zog einen langen Feuerstreifen über den Himmel.

„Jetzt stirbt jemand!“ sagte die Kleine; denn die alte Großmutter, welche sie allein freundlich behandelt hatte, jetzt aber längst tot war, hatte gesagt: „Wenn ein Stern fällt, steigt eine Seele zu Gott empor!“

Sie strich wieder ein Schwefelholz gegen die Mauer; es warf einen weiten Lichtschein rings umher und im Glanze desselben stand die alte Großmutter hell beleuchtet mild und freundlich da.

„Großmutter!“ rief die Kleine, „o nimm mich mit dir! Ich weiß, daß du verschwindest, sobald das Schwefelholz ausgeht, verschwindest, wie der warme Kachelofen, der köstliche Gänsebraten und der große flimmernde Weihnachtsbaum!“ Schnell strich sie den ganzen Rest der Schwefelhölzer an, welche sich noch im Schächtelchen befanden, sie wollte die Großmutter festhalten; und die Schwefelhölzer verbreiteten einen solchen Glanz, daß es heller war als am lichten Tage. So schön, so groß war die Großmutter nie gewesen; sie nahm das kleine Mädchen auf ihren Arm und hoch schwebten sie empor in Glanz und Freude; Kälte, Hunger und Angst wichen von ihm — sie waren bei Gott.

Aber im Winkel am Hause saß in der kalten Morgenstunde das kleine Mädchen mit roten Wangen, mit Lächeln um den Mund — tot, erfroren am letzten Tage des alten Jahres. Der Morgen des neuen Jahres ging über der kleinen Leiche auf, welche mit den Schwefelhölzern, wovon fast ein Schächtelchen verbrannt war, dasaß. „Sie hat sich wärmen wollen!“ sagte man. Niemand wußte, was sie Schönes gesehen hatte, in welchem Glanze sie mit der alten Großmutter zur Neujahrsfreude eingegangen war.

%\drawchicken
\directlua{}
\end{document}
