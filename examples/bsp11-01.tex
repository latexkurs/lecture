\documentclass[]{beamer}

\usepackage[T1]{fontenc}
\usepackage[utf8]{inputenc}
\usepackage[ngerman]{babel}

%% allgemeine Themes
%\usetheme{AnnAbor}
%\usetheme{Antibes}
%\usetheme{Bergen}
%\usetheme{Berkeley}
%\usetheme{Berlin}
%\usetheme{Boadilla}
%\usetheme{CambridgeUS}
%\usetheme{Copenhagen}
%\usetheme{Darmstadt}
%\usetheme{Dresden}
%\usetheme{Frankfurt}
%\usetheme{Goettingen}
%\usetheme{Hannover}
%\usetheme{Ilmenau}
%\usetheme{JuanLesPins}
%\usetheme{Luebeck}
%\usetheme{Madrid}
%\usetheme{Malmoe}
%\usetheme{Marburg}
%\usetheme{Montpellier}
%\usetheme{PaloAlto}
%\usetheme{Pittsburgh}
%\usetheme{Rochester}
%\usetheme{Singapore}
%\usetheme{Szeged}
%\usetheme{Warsaw}

%% Color Themes
%\usecolortheme{albatross}
%\usecolortheme{beaver}
%\usecolortheme{beetle}
%\usecolortheme{crane}
%\usecolortheme{dolphin}
%\usecolortheme{dove}
%\usecolortheme{fly}
%\usecolortheme{lily}
%\usecolortheme{orchid}
%\usecolortheme{rose}
%\usecolortheme{seagull}
%\usecolortheme{seahorse}
\usecolortheme{whale}
%\usecolortheme{wolverine}
%\usecolortheme[named=green]{structure}

%% Outer Themes
%\useoutertheme{infolines}
%\useoutertheme{miniframes}
%\useoutertheme{smoothbars}
%\useoutertheme{sidebar}
%\useoutertheme{split}
%\useoutertheme{shadow}
%\useoutertheme{tree}
%\useoutertheme{smoothtree}

%% Inner Themes
%\useinnertheme{circles}
%\useinnertheme{rectangles}
%\useinnertheme{rounded}
%\useinnertheme{inmargin}

%% Font Themes
%\usefonttheme{serif}
%\usefonttheme{structurebold}
%\usefonttheme{structureitalicserif}
\usefonttheme{structuresmallcapsserif}

\begin{document}

	\title{Doller Vortrag}
	\author{Hans Wurst}
	\frame{\titlepage}				% wie \maketitle
	\frame{\tableofcontents}		% Folie mit Inhaltsverzeichnis

	\begin{frame}{Erste Folie}
		Inhalt der ersten Folie
	\end{frame}
	
	\section{Overlays}
	\begin{frame}{Liste mit Overlays}
		% Overlay für die gesamte Liste:
		\begin{itemize}[<+->]
			\item erster Listenpunkt
			\item zweiter Listenpunkt
			\item dritter Listenpunkt
		\end{itemize}
		% Overlay für jeden einzelnen Punkt:
		\begin{itemize}
			\item<+> erster Listenpunkt
			\item<+-> zweiter Listenpunkt % wird gar nicht angezeigt
			\item<+> dritter Listenpunkt
		\end{itemize}
	\end{frame}
	
	\subsection{bla}

	\begin{frame}{only, onslide, uncover}
		\only<1>{dieser Satz steht nur auf der ersten Folie\\}%
		\onslide<2>{dieser Satz steht nur auf der zweiten Folie\\}
		\uncover<3>{dieser Satz steht nur auf der dritten Folie\\}
	\end{frame}
	
	\section{Blocks}
	\begin{frame}[t]{eine Folie mit Blocks}
		\begin{block}{erster Block}
			Inhalt
		\end{block}
		\begin{block}{zweiter Block}<2>
			Inhalt
		\end{block}
		\begin{block}{dritter Block}
			Inhalt
		\end{block}
	\end{frame}

\end{document}
