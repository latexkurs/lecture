\documentclass%[fleqn]% ← linksbündige Formeln
{scrartcl}

\usepackage[T1]{fontenc}
\usepackage[utf8]{inputenc}
\usepackage[ngerman]{babel}

\usepackage{
	amsmath,
	amssymb,
	icomma,
	mathtools,
}

% Definition eigener Operatoren
\usepackage{amsopn}
\DeclareMathOperator{\myop}{megasin}


\begin{document}

\section{Eigene Befehle Definieren}

\newcommand{\meinersterbefehl}{Jedes Mal wenn der Befehl ausgeführt wird, erscheint im Dokument dieser Text.}

\meinersterbefehl
\meinersterbefehl

\renewcommand{\meinersterbefehl}{Ich kann es mir auch noch anders überlegen und stattdessen diesen Text ausgeben.}

\meinersterbefehl
\meinersterbefehl

\subsection{Befehle mit Argumenten}

\newcommand{\molekuel}[3][X]{Das Molekül #1$_{#2}$#3}

\molekuel{2}{P}
\molekuel[Co]{7}{O}





\section{Mathe}
\subsection{Mathe-Umgebungen}

% inline-math:
$ f(x) = n_0 + n_1 \cdot x + n_2 \cdot x^2 + n_3 \cdot x^3 + \cdots = \sum_{i=0}^\infty n_i \cdot x^i $

% display-math
\[
	f(x) = n_0 + n_1 \cdot x + n_2 \cdot x^2 + n_3 \cdot x^3 + \cdots = \sum_{i=0}^\infty n_i \cdot x^i
\]

\begin{displaymath}
	f(x) = n_0 + n_1 \cdot x + n_2 \cdot x^2 + n_3 \cdot x^3 + \cdots = \sum_{i=0}^\infty n_i \cdot x^i
\end{displaymath}

\begin{equation}
	f(x) = n_0 + n_1 \cdot x + n_2 \cdot x^2 + n_3 \cdot x^3 + \cdots = \sum_{i=0}^\infty n_i \cdot x^i
\end{equation}

% an & ausgerichtete Formeln (mit Nummerierung):
\begin{align}
	f(x) &= n_0 + n_1 \cdot x + n_2 \cdot x^2 + n_3 \cdot x^3 + \cdots \\
	     &= \sum_{i=0}^\infty n_i \cdot x^i
\end{align}

% an & ausgerichtete Formeln (ohne Nummerierung):
\begin{align*}
	f(x) &= n_0 + n_1 \cdot x + n_2 \cdot x^2 + n_3 \cdot x^3 + \cdots \\
	     &= \sum_{i=0}^\infty n_i \cdot x^i
\end{align*}

\[ x_{n+1} = x_n - \frac{f(x_n)}{f’(x_n)} \]
\[ x_{n+1} = x_n - \frac{f(x_n)}{f^\prime (x_n)} \]




\subsection{Griechische Buchstaben}

\begin{align*}
\alpha        \theta        o           \tau         \\
\beta         \vartheta    \pi          \upsilon     \\
\gamma        \gamma       \varpi       \phi         \\
\delta        \kappa       \rho         \varphi      \\
\epsilon      \lambda      \varrho      \chi         \\
\varepsilon   \mu          \sigma       \psi         \\
\zeta         \nu          \varsigma    \omega       \\
\eta          \xi                                    \\
                                                     \\
\Gamma        \Lambda      \Sigma       \Psi         \\
\Delta        \Xi          \Upsilon     \Omega       \\
\Theta        \Pi          \Phi
\end{align*}

% Zum finden aller möglicher Symbole:
% http://detexify.kirelabs.org/classify.html


\subsection{Operatoren}

\begin{equation}
	\sin(x)
	\cos(y)
	\lim_{\phi\rightarrow 0} \tan(\phi)
\end{equation}

% Eigene Operatoren können in der Präambel definiert werden
\begin{equation}
	\myop(x)
\end{equation}



\subsection{Matrizen}

\begin{align}
	A &= 
 		\begin{pmatrix}
 			a & b \\
	 		c & d \\
 		\end{pmatrix} \\
	\det(A) &= 
		\begin{vmatrix}
 			a & b \\
	 		c & d \\
 		\end{vmatrix}
\end{align}

\end{document}