% \iffalse meta-comment
% 
% Copyright (C) 2015 by Lieschen Müller
% 
% This file may be distributed and/or modified under the
% conditions of the LaTeX Project Public License, either version 1.3
% of this license or (at your option) any later version.
% The latest version of this license is in:
% 
%     http://www.latex-project.org/lppl.txt
% 
%
% Do not distribute a modified version of this file under the same name.
% 
% \fi
% \iffalse
% 
%<*driver>
\ProvidesFile{meinpaket.dtx}[2016/02/02 v0.1 Source für dolles Paket]
%</driver>
%<package>\NeedsTeXFormat{LaTeX2e}[2007/07/20]
%<package>\ProvidesPackage{meinpaket}[2016/02/05 v0.1 Dolles Paket]
%
%<*batchfile>
\begingroup
\input{docstrip.tex}

\keepsilent

\declarepreamble\package

Copyright (C) 2015 by Lieschen Müller

This file may be distributed and/or modified under the
conditions of the LaTeX Project Public License, either version 1.3
of this license or (at your option) any later version.
The latest version of this license is in:

    http://www.latex-project.org/lppl.txt


Do not distribute a modified version of this file under the same name.


\endpreamble
\postamble

This work consists of the file  meinpaket.dtx
             the derived files  meinpaket.sty
                           and  meinpaket.pdf

\endpostamble

\askforoverwritefalse
\generate{\file{meinpaket.sty}{\from{meinpaket.dtx}{package}\usepreamble\package}}
\endgroup
%</batchfile>
%
%<*driver>
\documentclass{ltxdoc}
\EnableCrossrefs
\CodelineNumbered
\RecordChanges
\usepackage[ngerman]{babel}
\usepackage[T1]{fontenc}
\usepackage[utf8]{inputenc}
\begin{document}
\DocInput{meinpaket.dtx}
\end{document}
%</driver>
% \fi
% 
% \CheckSum{4}
% 
% \changes{v0.1}{2015/02/05}{Initial version}
% 
% \GetFileInfo{meinpaket.dtx}
% 
% \title{Mein Paket\thanks{Diese Anleitung bezieht sich auf Version \fileversion}}
% \author{Lieschen Müller}
% \date{\filedate}
% \maketitle
% 
% \begin{abstract}
%   \noindent Dieses tolle Paket tut tolle Dinge.
% \end{abstract}
% 
% \tableofcontents
% 
% \section{Anleitung}
% So funktioniert mein tolles Paket …
% 
% \section{Implementierung}
% So habe ich mein Paket implementiert:
% 
% \iffalse
%<*package> 
% \fi
%    \begin{macrocode}
\providecommand{\meinbefehl}{Hier steht der eigentliche Inhalt des Pakets}
%    \end{macrocode}
% \iffalse 
%</package>
% \fi
% 
% \section{Bekannte Fehler}
% \begin{itemize}
%   \item keine bekannt
% \end{itemize}
% 
% \Finale
% 
% \obeyspaces
% \typeout{**************************************************************}
% \typeout{*                                                            *}
% \typeout{* To finish the installation you have to move the following  *}
% \typeout{* file into a directory where TeX can find it:               *}
% \typeout{*                                                            *}
% \typeout{* meinpaket.sty                                              *}
% \typeout{*                                                            *}
% \typeout{* The documentation should have been produced                *}
% \typeout{* along with the other file.                                 *}
% \typeout{*                                                            *}
% \typeout{* Happy TeXing!                                              *}
% \typeout{*                                                            *}
% \typeout{**************************************************************}
% 
\endinput


