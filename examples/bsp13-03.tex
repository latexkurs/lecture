% !TEX program = lualatex
\documentclass[fontsize=12pt]{scrartcl}

\usepackage{polyglossia} 
\setmainlanguage{german}
\usepackage{amsmath}        % vor unicode-math laden
\usepackage{unicode-math}   % lädt auch fontspec
\usepackage{luacode}  


% Lua erzeugt Pseudozufallszahlen
\begin{luacode*}
math.randomseed(os.time())
a = math.random(5,20) 
b = math.random(2,10)
\end{luacode*}

% Ausgabe vereinfachen
\newcommand{\lprint}[1]{\directlua{tex.sprint(#1)}}

\begin{document}
	\begin{align*}
		&\text{Zufallszahlen:} &
			a &=\lprint{a}\\
			&& b &=\lprint{b}\\[1em]
		&\text{Summe:} &
			\lprint{a}+\lprint{b} &= \lprint{a+b}\\[1em]
		&\text{Differenz:} &
			\lprint{a}-\lprint{b} &=\lprint{a-b}\\[1em]
		&\text{Produkt:} &
			\lprint{a}\cdot\lprint{b} &= \lprint{a*b}\\[1em]
		&\text{Quotient:} &
			\frac{\lprint{a}}{\lprint{b}} 
			&= \lprint{math.floor(a/b)}\mod \luaexec{tex.print(a\%b)}\\
			&&&= \lprint{a/b}
	\end{align*}

\end{document}
              