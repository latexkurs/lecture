% file pi.tex version 0.993
%
% **** Compute Pi in TeX! ****
% 
%
% Author: D. Roegel (roegel@loria.fr)
%  
% Version 0.96: 22 July 1996 
%    First release
%
% Version 0.97: 22 July 1996
%    Modified by J. Gelinas (jacquesg@clic.net.ca)
%    Added one term to get correct last digits
%    Added a second optional argument: 
%       100    prints decimals 0..100
%       -2 100 prints only decimals 100..101
%    Print exact number of decimals
%
% Version 0.98: 23 July 1996
%    Modified by D. Roegel, so that the input -3 1 gives 141
%    and not 3.141. Also, some unnecessary braces inside \loop...\repeat
%    were removed. \ShowResult shortened by introduction of \NextDigit. 
% 
% Version 0.99: 23 July 1996
%    Modified by D. Roegel.
%    Improvements following suggestions by J. Gelinas (jacquesg@clic.net.ca).
%    Bug corrected, which made some digits false (\fontdimen\firstpos\xb=0pt
%    added in \updatefirstpos).
%    \N replaced by \count2 in \ShowResult, this enabling calls 
%    such as \ShowResult{1/\the\N...}{...}.
%
% Version 0.991: 23 July 1996
%    Modified by D. Roegel.
%    Two \UpdateFirstPos were removed.
%
% Version 0.992: 24 July 1996
%    Modified by D. Roegel, following several simplifications suggested
%    by J. Gelinas (jacquesg@clic.net.ca).
%    \Multiply removed, \Add and \Sub merged.
%    \ComputeArcTan shortened.
%
% Version 0.993: 24 July 1996
%    Modified by D. Roegel, following several simplifications suggested
%    by J. Gelinas (jacquesg@clic.net.ca).
%    \ComputeArcTan1/n computes now completely arctan(1/n).
%
%------------------------------------------------------------------------
% This programs uses the formula by John Machin:
%
%              Pi=16*arctan(1/5)-4*arctan(1/239)
%
% For arctan(x), we use the development
%
%  arctan(x)=\sum_{i=0}^\infty [{x^{2i+1}\over 2i+1} - {x^{2i+3}\over 2i+3}]
%
% One array (\xc) is used to store the partial sum up to {x^{2i+1}\over 2i+1}.
% A second array (\xa) us used to store the value of x^{2i+1}.
% A third array (\xb) is used to store {x^{2i+1}\over 2i+1}.
%
% The result is put in the array \xr.
%
% The implementation of arrays uses a trick shown by Tom Rokicki
% in his game of life program (life.tex).
%
% The number of digits you can compute depends on your implementation
% of TeX. I had no trouble computing 5000 digits. 
%
% The last digit can be wrong. And if it is a 0 or a 9, this decimal
% and previous ones can be wrong too. However, the absolute error is
% no greater than one unit of the last digit.
%
% If you want to know more about Pi, check the file sci.math.faq.
% See also http://www.primus.com/staff/paulp/useless/pi.html
% and http://www.tu-chemnitz.de/~arndt/joerg.html.
%
%------------------------------------------------------------------------
%
\newlinechar=`\^^J
\message{^^J***** Computation of Pi with John Machin's formula *****}
\message{^^J** i.e.: pi=16*arctan(1/5)-4*arctan(1/239)}
\message{^^JHow many decimals of pi do you want ? }

\read16to\nbdigits
\newcount\n
\n\nbdigits
\ifnum\n<0
  \multiply\n-1
  \message{^^JFirst decimal to output ? }
  \read16to\firstdigit
\else
  \advance\n1
  \def\firstdigit{0}
\fi
\newcount\lastdigit
\lastdigit\firstdigit
\advance\lastdigit\n
\advance\lastdigit-1

\newcount\index
\index\lastdigit
\advance\index11 
\divide\index4
\def\base{10000}
\def\basesp{10000sp}

\newcount\lastplusone
\lastplusone\index
\advance\lastplusone1

% \index is now the index for the last slot in the arrays

% slot 1 -> integer digits
% slot 2 -> digits 1 to 4
% slot 3 -> digits 5 to 8
% ...
% slot \index -> digits (\index-2) * 4 +1 to (\index-1) * 4 
%

\font\xa=cmr10 at 11truept % array for current values of (1/5)^{2n+1}
                           %                  and (1/239)^{2n+1}
\fontdimen\lastplusone\xa=0sp % this creates room
\font\xb=cmr10 at 13truept % array for current values of (1/5)^{2n+1}/(2n+1)
                           %                  and (1/239)^{2n+1}/(2n+1)
\fontdimen\lastplusone\xb=0sp % this creates room
\font\xc=cmr10 at 15truept % array for current sums of arctan(1/5)
                           %                    and arctan(1/239)
\fontdimen\lastplusone\xc=0sp % this creates room
\font\xr=cmr10 at 17truept % array for the result
\fontdimen\lastplusone\xr=0sp % this creates room

% (we have each time allocated one more slot than strictly necessary;
%  this avoids a test on \lastplusone in \updatefirstpos)

% \xa, \xb, \xc and \xr are now equal to 0

% Some variables (some of them are not strictly necessary, and might be
%  replaced by \count's):

\newcount\dv       % will hold dividers
\newcount\firstpos % first non empty slot
\newcount\I        % scratch register for loops
\newdimen\carry    % for carry (in additions) and borrows (in subtractions)
\newdimen\x        % a scratch variable
\newcount\N        % counts the terms
\newif\ifcont      % flag used to find when an operation on bignums is not done
\newcount\Sdv      % value of one digit (used in \ShowResult)
\newcount\dir      % toggle for alternating sums

% Initialization of working arrays

\def\InitializeArrays{
  {
  \I=1
  \loop
      \fontdimen\I\xa=0sp
      \fontdimen\I\xb=0sp
      \fontdimen\I\xc=0sp
      \advance\I1
    \ifnum\I<\lastplusone
  \repeat
  }
  }

% Initialization of the result

\newcount\I
\I=1
\loop
  \fontdimen\I\xr=0sp
  \advance\I1
  \ifnum\I<\lastplusone
\repeat


% divide array #1 by #2 beginning at slot \firstpos and up to \index;
% result is in array #3
% Maximum carry is 9999, so we need to be able to store 99999999.
% \dimen's can hold up to +/- 2,147,483,647 sp and
% \count's up to +/- 2,147,483,647, so it fits.

\def\Divide#1#2into#3{%
  \carry0sp
  \I\firstpos
  {
  \loop
       \x=\fontdimen\I#1
       \multiply\carry\base
       \advance\x\carry
       \carry\x
       \divide\x#2
       \fontdimen\I#3=\x
       \multiply\x#2
       \advance\carry-\x
       \advance\I1
     \ifnum\I<\lastplusone
  \repeat
  }
  }

% Add or Subtract #2 to #3, depending on #1. array #2 is not modified.

\def\Add#1#2to#3{%
  \carry0sp
  \I\index
  {
  \loop
      \x=\fontdimen\I#3
      \advance\x by #1\fontdimen\I#2
      \advance\x by #1\carry
      \fontdimen\I#3=\x
      \carry\x
      \divide\carry\base
      \multiply\carry\base
      \advance\x-\carry
      \divide\carry\base
      \ifdim\x<0sp
        \advance\x\basesp
        \advance\carry1sp
      \fi
      \fontdimen\I#3=\x
      \advance\I-1
      \ifnum\I<\firstpos \ifnum\carry=0 \contfalse
                         \else \conttrue \fi
      \else \conttrue
      \fi
    \ifcont
  \repeat
  }
  }


% Multiply array #1 by #2. Result is in #1.
% This macro is not used in this program, and only remains
% here for didactic and historical reasons.

\def\Multiply#1#2{%
  \carry0sp
  \I\index
  \loop
      \x=\fontdimen\I#1
      \multiply\x by #2
      \advance\x by \carry
      \fontdimen\I#1=\x
      \carry\x
      \divide\carry\base
      \multiply\carry\base
      \advance\x-\carry
      \fontdimen\I#1=\x
      \divide\carry\base
      \advance\I-1
      \ifnum\I<\firstpos \ifnum\carry=0 \contfalse 
                         \else \conttrue \fi
      \else \conttrue
      \fi
    \ifcont
  \repeat
  }

% update value of \firstpos; in the worst case, \firstpos
% gets increased by 2.

\def\updatefirstpos{
  \ifdim\fontdimen\firstpos\xa=0sp 
              \ifnum\firstpos<\lastplusone
                 \fontdimen\firstpos\xb=0sp
                 \advance\firstpos1 
              \fi
           \fi
  }

\def\UpdateFirstPos{\updatefirstpos\updatefirstpos}

% Compute arctan(1/#1). Arrays \xa and \xb are used. Result is in \xc.

\def\ComputeArcTan1/#1{%
  \firstpos1
  \InitializeArrays
  % initialize \xa with 1:
  \fontdimen1\xa=1sp
  \Divide\xa{#1}into\xa
  % now, \xa contains 1/#1
  \Add1\xa to\xb
  % \xb contains 1/#1
  \firstpos=2
  \dv=#1
  \multiply\dv\dv
  \N1
  \dir-1
  \message{^^JI am now computing the following sum: arctan(1/#1)=1/#1}
  \Add1\xb to\xc % first term
  \loop
       \Divide\xa\dv into\xa
       \UpdateFirstPos
     \ifnum\firstpos<\lastplusone
       \advance\N2
       \Divide\xa\N into\xb
       \ifnum\dir>0
         \message{+1/\the\N*1/#1^\the\N}
       \else
         \message{-1/\the\N*1/#1^\the\N}
       \fi
       \Add\dir\xb to\xc
       \multiply\dir-1
  \repeat
  }

% Extract the next digit from \count1

\def\NextDigit#1{
    \ifnum\count2<\lastdigit
       \Sdv\count1
       \divide\Sdv#1
       \advance\count2 1
       \ifnum\count2>\firstdigit
         \edef\res{\res\number\Sdv}
       \fi
       \multiply\Sdv#1
       \advance\count1-\Sdv
    \fi
  }

% Display the result; the digits are scanned one at a time,
% and those between \firstdigit and \lastdigit are extracted
% and saved in \res.

\def\ShowResult#1#2{{     
  \count0=2
  \count2=1
  \advance\lastdigit1
  \def\res{}
  \loop
      \count1=\fontdimen\count0#2
      \NextDigit{1000}
      \NextDigit{100}
      \NextDigit{10}
      \NextDigit{1}
      \advance\count0by 1
    \ifnum\count2<\lastdigit
  \repeat
  \advance\lastdigit-1  % for correct display in next line
  \message{^^J#1\res}
  }
}

\ComputeArcTan1/5

%\ShowResult{arctan(1/5)=0.}\xc

\message{^^JI multiply it by 4...}
\firstpos2
\Add1\xc to\xc  % alternative method is to write \Multiply\xc4
                % or to change \ComputeArcTan so that the initial value is 4
\Add1\xc to\xc  
\message{done}

% add 4*arctan(1/5) to \xr
\Add1\xc to\xr

%\ShowResult{4*arctan(1/5)=0.}\xr

\ComputeArcTan1/{239}

%\ShowResult{arctan(1/239)=0.}\xc


%\ShowResult{4*arctan(1/239)=0.}\xc

\message{^^JI subtract arctan(1/239) to 4*arctan(1/5)...}

\firstpos2
\Add{-1}\xc to\xr
\message{done}

\message{^^JAnd finally, I multiply it by 4 giving:}
\Add1\xr to\xr
\Add1\xr to\xr

\ifnum\firstdigit=0
  \ShowResult{pi[0 .. \number\lastdigit]=3.}\xr
\else
  \ShowResult{pi[\firstdigit.. \number\lastdigit]=}\xr
\fi

\bye

