% !TEX TS-program = xelatex
% !TEX encoding = UTF-8 Unicode
% !TEX spellcheck = de_DE
% 
% © 2015–2018 Moritz Brinkmann, CC-by-sa
% http://latexkurs.github.io

\documentclass[
	vorläufig=false, 
	blattnr=8,
	ausgabe=2021-12-15,
	abgabe=2021-12-22,
	lösung=true,
	shortverb,
]{../tex/latexkurs-exercise}


\begin{document}


\begin{aufgabe}[6]<2>{Bibliografien mit biblatex}
	In dieser Aufgabe sollen Sie ein Dokument (beliebigen Inhalts) erstellen, in das Sie eine Bibliografie 
	einfügen. Die Bibliografie soll mit dem Backend \pkg{biber}\footnote{Falls Sie \TeX works nutzen ist 
	|biber| nicht automatisch in der Programmauswahl enthalten. Sie können diesen Eintrag ergänzen, indem Sie 
	in den Einstellungen im Reiter \emph{Typeseting} ein weiteres Processing Tool hinzufügen.
	(Programmname |biber|, Argument |\$basename|)} 
	erstellt werden. 
	\begin{enumerate}[label=\alph*)]
		\item Legen Sie zunächst eine |.bib|-Datei an, in der sie mindestens drei Veröffentlichungen aus Ihrem 
		Fachgebiet\,/\,Studienfach einfügen. Sie können dafür gerne ein Literaturverwaltungsprogramm wie zum 
		Beispiel JabRef verwenden. Um sich möglichst wenig Arbeit zu machen, müssen Sie nicht alle Felder von 
		Hand füllen, sondern können einen fertigen BibTeX-Eintrag aus einer Internet-Datenbank verwenden.
		\item Wählen Sie nun einen – in Ihrem Fach üblichen – Zitierstil und laden Sie in einem |.tex|-
		Dokument das Paket \pkg{biblatex} mit den entsprechenden Optionen, die diesen Zitierstil aktivieren. 
		Konsultieren Sie gegebenenfalls die Paketdokumentation.
		\item Schreiben Sie nun einen kurzen Text in Ihr Dokument, in dem Sie die in der |.bib|-Datei 
		gesammelten Werke referenzieren. Spielen Sie ruhig ein wenig mit den verschiedenen Arten den |\cite|-
		Befehl aufzurufen herum.
		\item Sorgen Sie nun dafür, dass das Literaturverzeichnis in Ihrem Dokument auftaucht, indem Sie die 	
		entsprechenden Befehle aufrufen und die Referenzen von |biber| erstellen lassen.
		\item \emph{Bonusaufgabe:} Manchmal ist es notwendig, in einem Dokument mehrere getrennte 
		Bibliographien auszugeben, z.\,B. eine für Primär- und eine für Sekundärquellen. Erweitern Sie Ihr 
		Dokument so, dass zwei Bibliographien erstellt werden und \pkg{biblatex} Referenzen automatisch in 
		die richtige einordnet.
		\footnote{Wie Sie die unterschiedlichen Quellen markieren, sodass \pkg{biblatex} sie unterscheiden 
		kann, ist dabei Ihnen überlassen.}
	\end{enumerate}
	\abgabe{Fertiges Dokument, Quellcode und |.bib|-Datei als Zip im Moodle.}
\end{aufgabe}

\lösung{08_loesung_biblatex}

\begin{aufgabe}[6]{Mehrsprachiger Satz}
	\begin{enumerate}[label=\alph*)]
		\item Verfassen Sie einen Text, in dem Sie die Vor- und Nachteile des Sprachsupports in \LaTeX\ 
		darstellen. Würden Sie etwa einer befreundeten Linguistin oder einem befreundeten 
		Sprachwissenschaftler den Einsatz von \LaTeX\ für mehrsprachige Dokumente empfehlen?
		\item Verwenden Sie in Ihrem Dokument mindestens drei verschiedene Sprachen. Nutzen Sie dabei jeweils 
		die verschiedenen Funktionen zur Sprach-Umschaltung von \pkg{babel} oder \pkg{polyglossia}. Wenn Sie 
		eine Sprache studieren, dann sollte diese Sprache auf jeden Fall im Dokument vorkommen. 
		\item Falls Sie eine Sprache sprechen, die ein anderes als das lateinische Alphabet nutzt, probieren 
		Sie sie in Ihr Dokument einzufügen! Sorgen Sie dabei für eine korrekte Ausgabekodierung\footnote{Entweder durch den richtigen Einsatz des \pkg{fontenc}-Pakets, oder durch die Verwendung von 
		\hologo{XeLaTeX} oder \hologo{LuaLaTeX}} damit der eingegebene Text im PDF auch tatsächlich angezeigt 
		werden kann.
	\end{enumerate}
	\abgabe{Quellcode und fertiges Dokument als Zip im Moodle.}
\end{aufgabe}

\lösung{08_loesung_polyglossia}


\end{document}
