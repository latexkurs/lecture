% !TEX program = xelatex
% !TEX root = 06_umfangreiches_dokument.tex
% !TEX encoding = UTF-8 Unicode
% !TEX spellcheck = de_DE
% 
% © 2016 Moritz Brinkmann, CC-by-sa
% http://latexkurs.github.io


Bei der Aufteilung eines Umfangreichen Dokuments sollte man die folgenden Punkte beachten:
\begin{itemize}
	\item Es empfiehlt sich den Header (und ggf. eigene Definitionen) in eine eigene Datei auszulagern. Diese 
	sollte mit |\input| eingebunden werden, da sie bei jedem \LaTeX-Durchlauf, unabhängig von 
	|include|-Einstellungen benötigt wird.
	\item Jedes Kapitel sollte in eine \emph{eigene} Datei, die dann mit dem |\include|-Befehl eingebunden wird.
\end{itemize}

Eine mögliche Gestaltung der Ordnerstruktur wurde in der Vorlesung gezeigt:

\begin{center}
\begin{tikzpicture}[
	every node/.style={draw=black,thick,anchor=west},
	grow via three points={one child at (0.5,-0.7) and two children at (0.5,-0.7) and (0.5,-1.4)},
 	edge from parent path={(\tikzparentnode.south) |- (\tikzchildnode.west)}]
 	\ttfamily
	\node [fill=blue!30] {main/}
		child {node {main.tex}}
		child {node {header.tex}}
%		child {node {defs}}
		child {node [fill=blue!30] {bilder/}
			child {node {bild1.png}}
			child {node {bild2.jpg}}
		}
		child [missing] {}
		child [missing] {}
		child {node [fill=blue!30] {inhalte/}
			child {node {chapter1.tex}}
			child {node {chapter2.tex}}
			child {node {chapter3.tex}}
			child {node {chapter4.tex}}
		}
	;
\end{tikzpicture}
\end{center}

\clearpage