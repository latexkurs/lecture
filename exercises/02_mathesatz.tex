% !TEX TS-program = xelatex
% !TEX encoding = UTF-8 Unicode
% !TEX spellcheck = de_DE
% 
% © 2015–2018 Moritz Brinkmann, CC-by-sa
% http://latexkurs.github.io

\documentclass[
	vorläufig=false, 
	blattnr=2,
	ausgabe=2018-10-29,
	abgabe=2018-11-05,
	lösung=false,
	shortverb,
]{../tex/latexkurs-exercise}

\usepackage{mathtools}



\begin{document}


\begin{aufgabe}[6]{Maxwell-Gleichungen}
	Jeder Physiker sollte einmal im Leben die Maxwell-Gleichungen \TeX{}en, und auch für Studis anderer Fächer bieten diese Formeln eine gute Möglichkeit, den Mathesatz zu testen. Die vier Gleichungen lauten:                                                                                                           
                                                                                                                                                                                                                                                                                                                               
        \[\textstyle \nabla \times \vec E = -\frac{\partial \vec B}{\partial t}\quad \nabla \times \vec B = \vec j + \frac{\partial \vec E}{\partial t}\quad \nabla \cdot \vec E = \rho\quad                                                                                                                                   
        \nabla \cdot \vec B = 0\]                                                                                                                                                                                                                                                                                              
                                                                                                                                                                                                                                                                                                                               
   \noindent Überlegen Sie sich eine passende Formatierung und eine gute, übersichtliche Darstellung.\footnote{Die hier gezeigte Darstellung ist nicht optimal; sie sollen sich also explizit eine bessere überlegen.} Korrigieren Sie Abstände, falls nötig, wählen Sie eine gute und passende Schrift, überlegen Sie sich mögliche Auszeichnungsformen vektorieller Größen etc.                                                                                                                                                                                                                                                                       
                                                                                                                                                                                                                                                                                                                               
   Die nötigen Zeichen für diese Aufgabe finden Sie in der Datei |symbols-a4|\footnote{\texttt{texdoc symbols-a4}} oder mit dem \href{http://detexify.kirelabs.org/classify.html}{Detexify-Tool}.\footnote{\url{http://detexify.kirelabs.org/classify.html}}                                                              
                                                                                                                                                                                                                                                                                                                               
   \abgabe{Den Quelltext per Mail und als Ausdruck.}
\end{aufgabe}

\lösung{02_loesung_maxwell}

\begin{aufgabe}[3]{Integrale}
   Ein befreundeter Mathematiker hat eine eine großartige neue Theorie aufgestellt, die mit folgendem Integral-Konstrukt notiert wird:                                                                                                                                                                                    
   \begin{center}                                                                                                                                                                                                                                                                                                         
        \includegraphics{02_komischeintegrale} % Das habt Ihr Euch wohl so gedacht ;-)                                                                                                                                                                                                                                  
   \end{center}                                                                                                                                                                                                                                                                                                           
   Helfen Sie Ihrem Mathematiker-Freund beim Setzen seiner Theorie und bauen Sie das seltsame Integral aus vorhandenen Integralsymbolen in \hologo{pdfLaTeX} nach.                                                                                                                                                        
   
	\tipp{Suchen Sie in |symbols-a4| nach |txfonts/pxfonts| und benutzen Sie den \texttt{\textbackslash kern}-Befehl um horizontale Abstände zu verändern.}
   \abgabe{Den Quelltext per Mail, das fertige PDF als Ausdruck.} 
\end{aufgabe}

\lösung{02_loesung_integrale}

\begin{aufgabe}[3]{Fallunterscheidung mit Cases}
	Die Umgebung |cases| ermöglicht den Satz von Fallunterscheidungen:                                                                                                                                                                                                                                                     	\begin{lstlisting}                                                                                                                                                                                                                                                                                                             
\[ a =                                                                                                                                                                                                                                                                                                                         
  \begin{cases}                                                                                                                                                                                                                                                                                                                
    b & b > 0 \\                                                                                                                                                                                                                                                                                                         
    -b & b < 0\\                                                                                                                                                                                                                                                                                                               
    0 & b = 0                                                                                                                                                                                                                                                                                                                  
  \end{cases}                                                                                                                                                                                                                                                                                                                  
\]                                                                                                                                                                                                                                                                                                                             
\end{lstlisting} 
   Das \emph{muss} zwar im Mathesatz passieren, kann aber auch für Textinhalte (im Mathemodus mit \texttt{\textbackslash text\{\meta{normaler Text}\}} zu setzen) nützlich sein.
   
   Alternativ kann man auch eine |matrix|-Umgebung verwenden und die nötige Klammer von Hand (z.\,B. \texttt{\textbackslash left(}) setzen. Verwenden Sie nun letzteres, um zwei Formeln aus dem tractatus logico-philosophicus in folgender Form zu setzen:                                                                                                                                                                                                                            
     \begin{displaymath}                                                                                                                                                                                                                                                                                                    
     \text{tractatus logico-philosophicus, Satz 6.}                                                                                                                                                                                                                                                                        		\begin{cases}                                                                                                                                                                                                                                                                                                  
        03:\hspace*{0.65em} [o,\xi, \xi +1]\\                                                                                                                                                                                                                                                                  
     		231:\ 1+1+1+1 = (1+1) + (1+1)                                                                                                                                                                                                                                                                          
     	\end{cases}                                                                                                                                                                                                                                                                                                    
   \end{displaymath}                                                                                                                                                                                                                                                                                                      
	\abgabe{Den Quelltext per Mail, das fertige PDF als Ausdruck.}
\end{aufgabe}

\lösung{02_loesung_fallunterscheidung}



\end{document}
