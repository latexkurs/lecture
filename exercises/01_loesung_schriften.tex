% !TEX program = xelatex
% !TEX root = 01_schriften_kodierungen.tex
% !TEX encoding = UTF-8 Unicode
% !TEX spellcheck = de_DE
% 
% © 2016 Moritz Brinkmann, CC-by-sa
% © 2017-2022 Maximilian Jalea, CC-by-sa
% http://latexkurs.github.io


\begin{enumerate}[label=\alph*)]
		\item Unter \hologo{pdfLaTeX} kann man Schriften durch Laden der entsprechenden Pakete nutzen:

\begin{lstlisting}
\documentclass{minimal}

\usepackage[ngerman]{babel}
\usepackage{tgpagella}
\usepackage[T1]{fontenc} % nicht zwingend notwendig
\usepackage{blindtext}

\begin{document}

\blindtext

\end{document}
\end{lstlisting}


		\item Mit \hologo{XeLaTeX} oder \hologo{LuaLaTeX} sollte fontspec verwendet werden: 

\begin{lstlisting}
\documentclass{scrartcl}
\usepackage{polyglossia, xltxtra} % bzw. fontspec für LuaLaTeX
\setmainlanguage{german}

\setmainfont{Linux Libertine O}
\setsansfont{Linux Biolinum O}

\begin{document}
  \section{Serifenlose eignen sich gut für Überschriften}
  Serifen sollen das Lesen erleichtern, indem Sie den Augen als Linie dienen. Deshalb eignen sich Serifenschriften besonders gut als Brotschrift.
\end{document}
\end{lstlisting}

\end{enumerate}


