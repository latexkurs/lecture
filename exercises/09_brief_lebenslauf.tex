% !TEX TS-program = xelatex
% !TEX encoding = UTF-8 Unicode
% !TEX spellcheck = de_DE
% 
% © 2015–2018 Moritz Brinkmann, CC-by-sa
% © 2019-2022 Maximilian Jalea, CC-by-sa
% http://latexkurs.github.io

\documentclass[
	vorläufig=false, 
	blattnr=9,
	ausgabe=2022-12-07,
	abgabe=2022-12-22,
    solutions=true,
	shortverb,
]{../tex/latexkurs-exercise}

\usepackage[output-decimal-marker={,}]{siunitx}


\begin{document}

\begin{abstract}
\noindent Dies ist der letzte Übungszettel im \LaTeX-Kurs und damit Ihre letzte Gelegenheit Punkte für das Bestehen der Veranstaltung zu sammeln.
Sie haben die Vorlesung erfolgreich absolviert, wenn Sie insgesamt \num{60.0} oder mehr Punkte haben.
\end{abstract}

\noindent Ein bekannter Verlag sucht für hochwertige Veröffentlichungen einen Kenner des Satzsystemes \LaTeX. 
Da Sie gerade auf der Suche nach einem neuen Arbeitsplatz sind, sollten Sie diese Chance ergreifen und sich für die Stelle melden.

\begin{aufgabe}[6]{Anschreiben}
	Kontaktieren Sie zunächst den Verlag mit einem Motivationsschreiben, also einem Brief an den Empfänger, 
	der kurz begründet, warum Sie hervorragend geeignet sind.
	
	Da es um die Suche nach einem \LaTeX-Spezialisten geht, sollte der Brief formal für sich sprechen. Auf 
	den Inhalt müssen Sie also keinerlei Wert legen, aber das Aussehen sollte Ihren erfahrenen Umgang mit
	\LaTeX\ zeigen.

	Erstellen Sie also einen Brief, der Absender, Empfänger, Datum, Titel, Anrede etc. enthält. Außerdem soll 
	eine Anlage (siehe Aufgabe 2) angezeigt werden. Verwenden Sie die Klasse |scrlttr2| aus
	dem KOMA-Bundle. Sollte Ihnen eine andere Briefklasse lieber sein, können Sie diese verwenden, wenn Sie
	deren Vorteile gegenüber |scrlttr2| ausführen.
	\abgabe{Quelltext und fertiges Dokument als Zip per Moodle.}
\end{aufgabe}

\lösung{09_loesung_anschreiben}


\begin{aufgabe}[6]{Lebenslauf}\label{lebenslauf}
	Neben dem Motivationsschreiben gehört zu einer Bewerbung selbstverständlich ein Lebenslauf. Fertigen Sie 
	einen solchen an unter Verwendung einer der in der Vorlesung besprochenen Klassen. Hier gilt ebenfalls: 
	Das Dokument spricht für sich selbst. Sie müssen also dem Inhalt keine Beachtung schenken und können 
	beliebig (un-)sinnige Sachen schreiben.
	\abgabe{Quelltext und fertiges Dokument als Zip per Moodle.}
\end{aufgabe}

\lösung{09_loesung_lebenslauf}


\end{document}
