% !TEX program = xelatex
% !TEX root = 03_tabellen.tex
% !TEX encoding = UTF-8 Unicode
% !TEX spellcheck = de_DE
% 
% © 2016 Moritz Brinkmann, CC-by-sa
% http://latexkurs.github.io

Eine Möglichkeit die Tabelle zu setzen wäre zum Beispiel:

\begin{LTXexample}[pos=b]
\begin{table}
  \centering
  \begin{tabular}{llcccl}
    \toprule
    && \multicolumn{2}{c}{\textbf{Saison}} \\
    \cmidrule{3-4}
    \textbf{Produkt} & \textbf{Herkunft} & \textbf{Beginn} & \textbf{Ende} & \textbf{Handelskl.} & \textbf{Verfügbarkeit}\\
    \midrule
    Auberginen & Frankreich & Juli & September & I &  \\
    Esskastanien & Frankreich & \multicolumn{2}{c}{September} & I &  \\
    Feldsalat & Deutschland & Oktober & Februar & II & ja \\
    Kürbis & Deutschland & August & Dezember & I & ja\\
    Rote Beete & Italien & September & Februar & I  & ja\\
    Zucchini & Spanien & Juni & Oktober & II &  \\
    Zwiebeln & Deutschland & Mai & Oktober &  & \\
    \bottomrule
  \end{tabular}
  \caption{Unser saisonales Gemüseangebot}
\end{table}
\end{LTXexample}