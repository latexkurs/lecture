% !TEX program = xelatex
% !TEX root = 05_diagramme.tex
% !TEX encoding = UTF-8 Unicode
% !TEX spellcheck = de_DE
% 
% © 2016 Moritz Brinkmann, CC-by-sa
% © 2018-2022 Maximilian Jalea, CC-by-sa
% http://latexkurs.github.io

Daten aus externen Dateien lassen sich in \pkg{pgfplots} mit dem |table|-Befehl einbinden. Eine Mögliche Darstellung wäre die folgende:

\begin{LTXexample}[pos=b,rframe={},preset=\centering]
\begin{tikzpicture}
  \begin{axis}[
    xlabel={$t [\si{s}]$},
    ylabel={Anzahl Zerfälle},
    xmin=-10,
    xmax=610,
    legend cell align=left,
  ] 
    \addplot [
      blue,
      only marks,
      mark=.,
      error bars/.cd,
      y dir=both,
      y fixed relative=.1,
    ] table [
      x=zeit,
      y=zerfaelle,
    ] {05_messwerte.dat};
  \end{axis}
\end{tikzpicture}
\end{LTXexample}

Für exponentielle Zusammenhänge eignet sich besonders eine logarithmische Darstellung, die man mit |semilogyaxis| anstatt |axis| erhält:

\begin{center}
\begin{tikzpicture}
  \begin{semilogyaxis}[
    xlabel={$t [\si{s}]$},
    ylabel={Anzahl Zerfälle},
    xmin=-10,
    xmax=610,
    legend cell align=left,
  ] 
    \addplot [
      blue,
      only marks,
      mark=.,
%      mark size=.5pt,
      error bars/.cd,
      y dir=both,
      y fixed relative=.1,
    ] table [
      x=zeit,
      y=zerfaelle,
    ] {05_messwerte.dat};
  \end{semilogyaxis}
\end{tikzpicture}
\end{center}

Mithilfe des Pakets \pkg{pgfplotstable} lassen sich aus den Tabellendaten leicht weitere Spalten errechnen. So kann man eine automatisch berechnete Ausgleichsgerade hinzufügen:

\begin{lstlisting}
\addplot table [y={create col/linear regression={y=zerfaelle}}, mark=none]
  {05_messwerte.dat};
\end{lstlisting}

\begin{center}
\begin{tikzpicture}
  \begin{semilogyaxis}[
    xlabel={$t [\si{s}]$},
    ylabel={Anzahl Zerfälle},
    xmin=-10,
    xmax=610,
    legend cell align=left,
  ] 
    \addplot [
      blue,
      only marks,
      mark=.,
%      mark size=.5pt,
      error bars/.cd,
      y dir=both,
      y fixed relative=.1,
    ] table [
      x=zeit,
      y=zerfaelle,
    ] {05_messwerte.dat};
    \addplot table [y={create col/linear regression={y=zerfaelle}}, mark=none]
  {05_messwerte.dat};
  \end{semilogyaxis}
\end{tikzpicture}
\end{center}
