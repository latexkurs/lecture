% !TEX program = xelatex
% !TEX root = 02_mathesatz.tex
% !TEX encoding = UTF-8 Unicode
% !TEX spellcheck = de_DE
% 
% © 2016 Moritz Brinkmann, CC-by-sa
% © 2022 Maximilian Jalea, CC-by-sa
% http://latexkurs.github.io

\newcommand\zeitpunkt[5]{#1.#2.#3 #4:#5}

\begin{enumerate}[label=\alph*)]
\item Es war ein Makro wie das Folgende gefordert. Die Formatierung war dabei beliebig (z.\,B. \zeitpunkt{05}{10}{2021}{19}{32} oder 5. Oktober 2021 um 19\,Uhr und 32\,Minuten).
\begin{lstlisting}
\newcommand\zeitpunkt[5]{#1.#2.#3 #4:#5}
\end{lstlisting}
Da dieses Makro in Tabellen verwendet werden soll, bietet es sich an, auf einen Zusatz wie \emph{Uhr} und ausgeschriebene Monatsnamen zu verzichten.
\item In dieser Aufgabe war eine Tabelle wie die folgende gefordert:
\begin{LTXexample}[pos=t,preset=\centering]
\begin{table}
  \centering
  \caption{Geburten im Zoo \TeX -Hausen}
  \label{tab:geburten}
  \begin{tabular}{lll}
    \toprule
    {\textbf{Tierart}}& {\textbf{Name}} &  {\textbf{Geburtszeitpunkt}}    \\
    \midrule 
    Elefant           & Getrude         & \zeitpunkt{03}{04}{2020}{04}{20}\\
    Tiger             & Streif          & \zeitpunkt{23}{10}{2020}{18}{43}\\
    Hamster           & Kringel         & \zeitpunkt{15}{01}{2021}{09}{04}\\
    \bottomrule
  \end{tabular}
\end{table}
\end{LTXexample}

\pagebreak
\item Nun sollte man die Tabelle um eine Gewichtsangabe erweitern. \pkg{siunitx} bietet mit dem Spaltentyp |S| eine praktische Hilfestellung:\\[.1ex]

\begin{LTXexample}[pos=b,preset=\centering]
\begin{table}
  \centering
  \caption{erweiterte Geburten im Zoo \TeX -Hausen}
  \label{tab:mehrgeburten}
  \begin{tabular}{lllS}
    \toprule
    {\textbf{Tierart}}&{\textbf{Name}}&{\textbf{Geburtszeitpunkt}}&\textbf{Gewicht} [\si{\kilo\gram}]\\
    \midrule 
    Elefant           & Getrude & \zeitpunkt{03}{04}{2020}{04}{20} & 100    \\
    Tiger             & Streif  & \zeitpunkt{23}{10}{2020}{18}{43} & 1.21   \\
    Hamster           & Kringel & \zeitpunkt{15}{01}{2021}{09}{04} & 0.003  \\
    \bottomrule
  \end{tabular}
\end{table}
\end{LTXexample}


\item Das Paket \pkg{siunitx} fasst jede Eingabe in der Form |364(5)| oder |97+-3| als Wert mit Fehler auf. Will man in der Ausgabe statt z.\,B. \num{97(3)} den Fehler durch $\pm$ getrennt haben (\num[separate-uncertainty]{97(3)}), muss man Tabellen-Definition entsprechen verändern:
\begin{lstlisting}
\begin{tabular}{lllS[separate-uncertainty]}
\end{lstlisting}

\pagebreak
\item Die |table|-Um\-ge\-bung „fängt“ Fußnoten ein. D.\,h., dass sie nur innerhalb der \verb|table|-Um\-ge\-bung existieren und daher nicht richtig gesetzt werden können.
Es gibt zahlreiche Möglichkeiten, dieses Problem zu vermeiden. Eine ist, die Tabelle in einer Minipage zu setzen. Dabei werden die Fußnoten mit einer internen Nummerierung direkt unter die Tabelle gesetzt.

Alternativ kann man die Befehle |\footnotemark| und |\footnotetext| verwenden, bei denen man allerdings die Nummerierung manuell vornehmen muss.
\begin{lstlisting}
\begin{minipage}\textwidth
  \captionof{table}{Überschrift für Tabelle in einer Minipage}
  \begin{center}
    \begin{tabular}{l}
      \toprule
      Tabellenkopf	\\
      \midrule
      Inhalt mit Fußnote\footnote{Fußnote in einer Tabelle} \\
      Mehr Inhalt\footnote{Noch eine Fußnote} \\
      Zeile mit \verb|\footnotemark|\footnotemark	\\
      \bottomrule
    \end{tabular}
  \end{center}
  \footnotetext[3]{Fußnote mit \verb|\footnotetext|}
\end{minipage}
\end{lstlisting}

\end{enumerate}
