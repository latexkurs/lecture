% !TEX TS-program = xelatex
% !TeX root = ./02_mathesatz.tex
% !TEX encoding = UTF-8 Unicode
% !TEX spellcheck = de_DE
% !TeX
% 
% © 2016 Moritz Brinkmann, CC-by-sa
% http://latexkurs.github.io

Das Integral-Konstrukt kann mit folgendem Code erzeugt werden:                                                                                                                                                                                                                                                                 
\begin{lstlisting}                                                                                                                                                                                                                                                                                                             
\documentclass{minimal}                                                                                                                                                                                                                                                                                                        
  \usepackage{amsmath,pxfonts}                                                                                                                                                                                                                                                                                                 
\begin{document}                                                                                                                                                                                                                                                                                                               
  \begin{displaymath}                                                                                                                                                                                                                                                                                                          
    \oint\kern.15em\vec\varoiintctrclockwise\kern-2.60em\varoiintclockwise                                                                                                                                                                                                                                                     
  \end{displaymath}                                                                                                                                                                                                                                                                                                            
\end{document}                                                                                                                                                                                                                                                                                                                 
\end{lstlisting}